
\documentclass[8pt]{beamer}

\usepackage[german]{babel}
\usepackage[utf8]{inputenc}
\usepackage{pdfpages}
\usepackage{color}
\usepackage{graphicx, import}
\usepackage{amsmath}
\usepackage{amssymb}
\usepackage{physics} % norm
\usepackage{tikz}
\usepackage{tkz-euclide}
\usepackage{enumitem}
\usepackage{pgfplots}
\usepackage{multicol}
\usepackage{tabularx}
\usepackage[numbers, square]{natbib}
\usepackage{mathtools}
\usepackage{transparent}
\usepackage{caption} % change style of figure 
\usepackage{subcaption}
\usepackage{csquotes} % automatic left quotation marks
\usepackage{booktabs}
\usepackage[noline]{algorithm2e}
\usepackage[super]{nth}
\usepackage{autonum} % auto equation numbering
\usepackage{bbm}

\captionsetup*[subfigure]{position=bottom}

\MakeOuterQuote{"}

\usetikzlibrary{calc, positioning,backgrounds, fit, patterns,shapes, snakes, chains, arrows, decorations.markings, arrows.meta}
%\tikzexternalize[prefix=out/figures/]
\newcolumntype{Y}{>{\centering\arraybackslash}X} % centered equidistant columns

\bibliographystyle{plainnat}
\usetheme[block=transparent]{metropolis}
%\usecolortheme{orchid}
%\setbeamercolor{block body}{use=structure,fg=black,bg=blue!20!white}
%\setbeamercolor{block title}{use=structure,fg=black,bg=blue!40!white}
\setbeamertemplate{frame footer}{\insertshortauthor\hfill\insertshortinstitute}
\setbeamercolor{footline}{fg=gray}

\DeclarePairedDelimiterX{\infdiv}[2]{(}{)}{%
	#1\;\delimsize\|\;#2%
}

\renewcommand{\epsilon}{\varepsilon}

\DeclareMathOperator*{\argmax}{arg\,max}
\DeclareMathOperator*{\argmin}{arg\,min}
\DeclareMathOperator{\JSD}{JS}
\DeclareMathOperator{\KL}{KL}
\newcommand{\T}{\mathrm{T}}
\newcommand{\R}{\mathbb{R}}
\newcommand{\cX}{\mathcal{X}}
\newcommand{\cG}{\mathcal{G}}
\newcommand{\cD}{\mathcal{D}}
\newcommand{\cF}{\mathcal{F}}
\newcommand{\cZ}{\mathcal{Z}}
\newcommand{\cB}{\mathcal{B}}
\newcommand{\N}{\mathbb{N}}
\newcommand{\E}{\mathbb{E}}
\newcommand{\fL}{\mathfrak{L}}

\makeatletter
\DeclareRobustCommand\onedot{\futurelet\@let@token\@onedot}
\def\@onedot{\ifx\@let@token.\else.\null\fi\xspace}

\def\eg{\emph{e.g}\onedot} \def\Eg{\emph{E.g}\onedot}
\def\ie{\emph{i.e}\onedot} \def\Ie{\emph{I.e}\onedot}
\def\cf{\emph{c.f}\onedot} \def\Cf{\emph{C.f}\onedot}
\def\etc{\emph{etc}\onedot} \def\vs{\emph{vs}\onedot}
\def\iid{\emph{i.i.d}\onedot} \def \Iid{\emph{I.i.d}\onedot}
\def\wrt{w.r.t\onedot} \def\dof{d.o.f\onedot}
\def\etal{\emph{et al}\onedot}
\makeatother

\newcommand{\pdata}{{p_{data}}}
\newcommand{\pmodel}{{p_{model}}}

%\theoremstyle{theorem}
%\newtheorem{lemma}[definition]{Lemma}
%\newtheorem{corollary}[definition]{Corollary}
%\newtheorem{remark}[definition]{Remark}
%\newtheorem{proposition}[definition]{Proposition}
%\theoremstyle{break}
%\newtheorem{theorem}[definition]{Theorem}
%\newtheorem{condition}[definition]{Condition}
\setbeamertemplate{section in toc}[sections numbered]
\setbeamertemplate{subsection in toc}{\leavevmode\leftskip=3.2em\rlap{\hskip-2em\inserttocsectionnumber.\inserttocsubsectionnumber}\inserttocsubsection\par}
\setbeamercolor{section in toc}{fg=black}
\setbeamercolor{subsection in toc}{fg=black}
\setbeamercolor{toc}{fg=black}

\AtBeginSection[]{}

\title[]{Das Ring Loading Problem}
%\subtitle{Seminar \glqq Mathematische Aspekte des maschinellen Lernens\grqq}
\author[Nikolas Klug]{Nikolas Klug}
\institute[Universität Augsburg]{Universität Augsburg}
\date{23. Dezember 2021}


\begin{document}
	{
	\setbeamertemplate{footline}{}
	\begin{frame}
		\titlepage
	\end{frame}
	}
	\addtocounter{framenumber}{-1}

	\begin{frame}{Quelle}
%		\begin{NoHyper}
%			\tableofcontents[]
%		\end{NoHyper}

		Hauptquelle:\\
		\emph{The Ring Loading Problem},\\
		 Alexander Schrijver, Paul Seymour, Peter Winkler, 1999,\\
		 \emph{SIAM Review 41 (4).}\\
	\end{frame}
	
	\begin{frame}{Problemstellung}
		Problem stammt aus Praxis...
	\end{frame}

	\section{Formalisierung}
	
	\begin{frame}{Formalisierung}
		\begin{columns}
			\column{0.6\linewidth}
			demands\\
			routings\\
			edge loads\\
			\column{0.4\linewidth}
			\centering
\begin{tikzpicture}[font=\scriptsize, node/.style={circle,thick,draw},
	l_2/.style={line width =0.25mm},
	scale=0.9, transform shape]
	\draw[l_2] (2,0) arc (0:360:2);	
	% equidistant points and arc
	\foreach \x [count=\p] in {0,...,7} {
		\node[shape=circle,fill=black, scale=0.5] (\p) at (\x*45-135:2) {};
	};
	\foreach \x [count=\p] in {0,...,7} {
		\draw (225 + \x*45:1.7) node {\p};
		%				\draw (-30-\x*60:2.4) node {$\bar{\p}$};
	}; 
%	\node (bottom) at (0, -2.8) {};
\end{tikzpicture}

		\end{columns} 
	\end{frame}

	\begin{frame}{Formalisierung}
		Ring Loading Problem
	\end{frame}

	\begin{frame}{NP-Vollständigkeit}
		Ring Loading ist NP-Vollständig + Beweis
	\end{frame}

	\begin{frame}{Relaxed Ring Loading}
		Relaxed Ring Loading
	\end{frame}
	
	\begin{frame}{Relaxed Ring Loading}
		Relaxed Ring Loading
	\end{frame}
	
	\begin{frame}{Approximations Algorithmus}
		Überblick über Approximations algo: 1. Bestimme "dünn-besetzte" Lösung von RRL 2. Reroute split demands 
	\end{frame}

	\begin{frame}{Minimale Lösungen}
		Definition Minimaler Lösungen
	\end{frame}

	\begin{frame}{Charakterisiung minimaler Lösungen}
		Theorem: nur wenige gesplittete demands	\\
		Definitionen crossing, parallel	
	\end{frame}
		
	\begin{frame}{RelaxedRingLoading mit Kapazitäten}
		Definition Kapazitäten und Problem mit Kapazitäten	
	\end{frame}
	
	\begin{frame}{Schnitte}
		Definition Schnitte, schneiden, demand across cut, cut condition
	\end{frame}

	\begin{frame}{Charakterisierung der Lösbarkeit}
		Lösbar genau dann wenn cut condition erfüllt + Beweis
	\end{frame}

	\begin{frame}{On the way to an algorithm...}
		Definition der Kapazitäten + einige Beobachtungen dazu.
	\end{frame}

	\begin{frame}{Einige Beobachtungen}
		Definition der Kapazitäten + einige Beobachtungen dazu.
	\end{frame}

	\begin{frame}{Der Algorithmus}
		Definition der Kapazitäten + einige Beobachtungen dazu.
	\end{frame}

	\section{Transformation in ein binäres Routing}
	\begin{frame}{Transformation in ein binäres Routing}
		Schrumpfen der Instanz
	\end{frame}

	\begin{frame}{Umrouten gesplitter Nachfragen}
		Änderung der Kantenlast
	\end{frame}

	\begin{frame}{Wahl des Umroutings}
		content...
	\end{frame}
	
	\begin{frame}{Approximationsgüte}
		content...
	\end{frame}
	
	\begin{frame}[allowframebreaks]
		\nocite{*}
		\bibliography{bibliography.bib}
	\end{frame}


\end{document}