% Mathematische Zeichens\"atze und Umgebungen
\usepackage{amsfonts, amssymb}	% Definition einer Liste mathematischer Fontbefehle und Symbole
\usepackage[intlimits,		% Integralgrenzen \"uber und unter dem Integral
]		% Summationsgrenzen \"uber und unter der Summe
           {amsmath}		% mathematische Verbesserungen
\usepackage{amsthm}		% spezielle theorem Stile
\usepackage{aliascnt} 

%-----------------------------------------------------------------------------------
% Hilfreiche Befehle
%-----------------------------------------------------------------------------------
\newcommand{\betrag}[1]{\lvert #1 \rvert}	        % Betrag
\newcommand{\norm}[1]{\lVert #1 \rVert}		        % Norm
\providecommand*{\Lfloor}{\left\lfloor}                 % gro\ss{}es Abrunden
\providecommand*{\Rfloor}{\right\rfloor}                % gro\ss{}es Abrunden
\providecommand*{\Floor}[1]{\Lfloor #1 \Rfloor}         % gro\ss{}es ganzes Abrunden
\providecommand*{\Ceil}[1]{\left\lceil #1 \right\rceil} % gro\ss{}es ganzes Aufrunden

\DeclareMathOperator{\e}{ex}
\DeclareMathOperator{\ma}{mate}
\DeclareMathOperator{\Ex}{Ex}

%-----------------------------------------------------------------------------------
%   Befehle f\"ur Nummerierung der Ergebnisse
%   fortlaufend innerhalb eines Abschnittes
%-----------------------------------------------------------------------------------
\theoremstyle{plain}            % normaler Stil
\newtheorem{theorem}{Theorem}[section]
% Lemma
\newaliascnt{lemma}{theorem}  
\newtheorem{lemma}[lemma]{Lemma}  
\aliascntresetthe{lemma}  
% Satz
\newaliascnt{satz}{theorem}  
\newtheorem{satz}[satz]{Satz} 
\aliascntresetthe{satz}
% Korollar
\newaliascnt{korollar}{theorem}  
\newtheorem{korollar}[korollar]{Korollar} 
\aliascntresetthe{korollar}
% Proposition
\newaliascnt{proposition}{theorem}  
\newtheorem{proposition}[proposition]{Proposition} 
\aliascntresetthe{proposition}
% Invariant
\newaliascnt{invariant}{theorem}
\newtheorem{invariant}[invariant]{Invariant}
\aliascntresetthe{invariant}
%-----------------------------------------------------------------------------------
\theoremstyle{definition}	% Definitionsstil
% Definition
\newaliascnt{definition}{theorem}  
\newtheorem{definition}[definition]{Definition} 
\aliascntresetthe{definition}
% Beispiel
\newaliascnt{beispiel} {theorem}  
\newtheorem{beispiel}[beispiel]{Beispiel} 
\aliascntresetthe{beispiel} 
% Problem
%\newaliascnt{problem}{theorem}  
%\newtheorem{problem}[problem]{Problem} 
%\aliascntresetthe{problem}
% Algorithmus
\newaliascnt{algorithmus}{theorem}  
\newtheorem{algorithmus}[algorithmus]{Algorithmus} 
\aliascntresetthe{algorithmus} 
%-----------------------------------------------------------------------------------
\theoremstyle{remark}		% Bemerkungsstil
% Bemerkung
\newaliascnt{bemerkung}{theorem}  
\newtheorem{bemerkung}[bemerkung]{Bemerkung}  
\aliascntresetthe{bemerkung} 
% Vermutung
\newaliascnt{vermutung}{theorem}  
\newtheorem{vermutung}[vermutung]{Vermutung}  
\aliascntresetthe{vermutung} 
% Notation
\newaliascnt{notation}{theorem}  
\newtheorem{notation}[notation]{Notation} 
\aliascntresetthe{notation}

%-----------------------------------------------------------------------------------
% automatische Referenzen mit interaktivem Text
%-----------------------------------------------------------------------------------

% Texte
\newcommand{\theoremautorefname}{Theorem}
\newcommand{\lemmaautorefname}{Lemma}
\newcommand{\satzautorefname}{Satz}
\newcommand{\korollarautorefname}{Korollar}
\newcommand{\propositionautorefname}{Proposition}
\newcommand{\invariantautorefname}{Invariant}

\newcommand{\definitionautorefname}{Definition}
\newcommand{\beispielautorefname}{Beispiel}
%\newcommand{\problemautorefname}{Problem}
\newcommand{\algorithmusautorefname}{Algorithmus}

\newcommand{\bemerkungautorefname}{Bemerkung}
\newcommand{\vermutungautorefname}{Vermutung}
\newcommand{\notationautorefname}{Notation}

%-----------------------------------------------------------------------------------
% Nummerierung der Gleichungen innerhalb der obersten Ebene
%-----------------------------------------------------------------------------------
\ifx\chapter\undefined 			% Kapitel sind definiert
  \numberwithin{equation}{section}	% Gleichungsnummern in Section
\else					% Kapitel sind nicht definiert
  \numberwithin{equation}{chapter}	% Gleichungsnummern in Kapiteln
\fi
