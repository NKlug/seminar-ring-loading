%-----------------------------------------------------------------------------------
% Kopf-Zeilen
%-----------------------------------------------------------------------------------

\usepackage[automark]{scrlayer-scrpage}	% Seiten-Stil f\"ur scrartcl
\pagestyle{scrheadings}		% Kopfzeilen nach scr-Standard		
\ifx\chapter\undefined 		% falls Kapitel nicht definiert sind
  \automark[subsection]{section}% Kopf- und Fusszeilen setzen
\else				% Kapitel sind definiert
  \automark[section]{chapter}	% Kopf- und Fusszeilen setzen
\fi

%-----------------------------------------------------------------------------------
%   Maske f\"ur \"Uberschrift 
%-----------------------------------------------------------------------------------
% Belegung der notwendigen Kommandos f\"ur die Titelseite
\newcommand{\autor}{Klug, Nikolas} 		% bearbeitender Student
\newcommand{\veranstaltung}{Seminar zur Optimierung und Spieltheorie} 	% Titel des ganzen Seminars
\newcommand{\matrikelnummer}{1474569}
\newcommand{\uni}{Institut f\"ur Mathematik der Universit\"at Augsburg} % Universit\"at
\newcommand{\lehrstuhl}{Diskrete Mathematik, Optimierung und Operations Research} % Lehrstuhl
\newcommand{\semester}{Wintersemester 21/22}	% Winter- oder Sommersemester mit Jahr
\newcommand{\datum}{16.12.2021} 			% Datumsangabe
\newcommand{\thema}{The Ring Loading Problem}  		% Titel der Seminararbeit

\newcommand{\ownline}{\vspace{.7em}\hrule\vspace{.7em}} % horizontale Linie mit Abstand

\newcommand{\seminarkopf}{	% Befehl zum Erzeugen der Titelseite 
 \textsc{\autor}  \hfill{\datum} \\ 
\textbf{\veranstaltung} \\ 
\uni \\ 
\lehrstuhl \\
\semester \hfill{Matrikelnummer: \matrikelnummer}
\ownline 

\begin{center}
{\LARGE \textbf{\thema}}
\end{center}

\ownline
}