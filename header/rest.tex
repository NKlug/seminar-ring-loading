\usepackage{array}		% erweiterte Tabellen

% Schriftzeichen, Format
\usepackage{latexsym}		% Latex-Symbole
\usepackage[latin1]{inputenc}	% Eingabekodierungen
\usepackage[english,ngerman]{babel}	% Mehrsprachenumgebung

% Layout
\usepackage{geometry}                    % Seitenränder
\geometry{a4paper, top=30mm, bottom=30mm, left=30mm, right=30mm}
\addtolength{\footskip}{-0.5cm}          % Seitenzahlen höher setzen
\usepackage{xcolor}                      % Farben


% Tabellen und Listen
\usepackage{float}		        % Gleitobjekte 
\usepackage[flushright]{paralist}       % Bessere Behandlung der Auflistungen

% Bilder
\usepackage[final]{graphicx}            % Graphiken einbinden

\usepackage{caption}                    % Beschriftungen
\usepackage{subcaption}                 % Beschriftungen f\"ur Unterteilung

%%%%%%
% Falls Zeichnungen mit pstricks erstellt werden sollen (Ausgabeprofil muss auf LaTeX -> PS -> PDF eingestellt werden)
%%%%%%
%\usepackage{pst-all}                    % Zeichnungen in Latex (kein pdflatex)
%\usepackage{pstricks-add}               % zus\"atzliches von pstricks
%\usepackage{pst-3dplot}                 % dreidimensionale Zeichnungen
%\usepackage{pst-eucl}                   % euklidisches Paket

\numberwithin{figure}{section}	% Abbildungsnummern in Section

%% My personal packages and config

\usepackage[super]{nth} % use superscripts for 1st, 2nd, 3rd
\usepackage{physics} % norm
\usepackage{bbm} % double struck numbers
\usepackage[activate=true,final,tracking=true,kerning=true,factor=1100,stretch=10,shrink=10]{microtype} % even better line spacing
\usepackage[capitalize,nameinlink,noabbrev]{cleveref} % better "\autoref"
\usepackage[colorinlistoftodos,prependcaption,textsize=tiny]{todonotes}
\usepackage{xargs} % Use more than one optional parameter in a new commands
\usepackage{enumitem} % changing enumeration styles
\usepackage[sort, numbers, square]{natbib} % citeauthor, citet
\usepackage{mathtools}
\usepackage{xspace} 

% Captions for figures
\captionsetup{justification=raggedright, format=plain, font=small,labelfont=bf}

\newcommandx{\unsure}[2][1=]{\todo[linecolor=orange,backgroundcolor=orange!25,bordercolor=orange,#1]{#2}}
\newcommandx{\Todo}[2][1=]{\todo[linecolor=yellow,backgroundcolor=yellow!25,bordercolor=yellow,#1]{#2}}
\newcommandx{\info}[2][1=]{\todo[linecolor=green,backgroundcolor=green!25,bordercolor=green,#1]{#2}}

% Centered, equally spaced columns
\newcolumntype{Y}{>{\centering\arraybackslash}X} % centered equidistant columns

\setenumerate{label=(\arabic*),itemsep=0mm} 

\renewcommand{\epsilon}{\varepsilon}

\DeclareMathOperator*{\argmax}{arg\,max}
\DeclareMathOperator*{\argmin}{arg\,min}
\DeclareMathOperator{\JSD}{JS}
\DeclareMathOperator{\KL}{KL}
\newcommand{\T}{\mathrm{T}}
\newcommand{\R}{\mathbb{R}}
\newcommand{\cX}{\mathcal{X}}
\newcommand{\cG}{\mathcal{G}}
\newcommand{\cD}{\mathcal{D}}
\newcommand{\cF}{\mathcal{F}}
\newcommand{\cO}{\mathcal{O}}
\newcommand{\cZ}{\mathcal{Z}}
\newcommand{\cB}{\mathcal{B}}
\newcommand{\N}{\mathbb{N}}
\newcommand{\E}{\mathbb{E}}
\newcommand{\Z}{\mathbb{Z}}
\newcommand{\fL}{\mathfrak{L}}
\newcommand{\RL}{\textsc{Ring Loading}\xspace}
\newcommand{\RRL}{\textsc{Relaxed Ring Loading}\xspace}

% boxed problem environment
\newenvironment{problem}[1]
{
%	\noindent\begin{fbox}
%		\begin{parbox}{\textwidth}
			\begin{center}
				\textsc{#1}
			\end{center}
}
{

%		\end{parsbox}
%	\end{fbox}
}
