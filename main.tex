\documentclass[paper=a4, 	% Seitenformat
		fontsize=11pt, 		% Schriftgr\"o\ss{}e
		abstract=true, 	% mit Abstrakt
		headsepline, 	% Trennlinie f\"ur die Kopfzeile
		notitlepage	% keine extra Titelseite
		]{scrartcl}

%%%%%%%%%%%%%%%%%%%%%%%%%%%%%%%%%%%%%%%%%%%%%%%%%%%%%%%%%%%%%%%%%%%%%%%%%%%%%%%%%%%%%%%%%%%
% Zusammenfassung einiger n�tzlicher Pakete und Befehle
%-----------------------------------------------------------------------------------
% Kopf-Zeilen
%-----------------------------------------------------------------------------------

\usepackage[automark]{scrlayer-scrpage}	% Seiten-Stil f\"ur scrartcl
\pagestyle{scrheadings}		% Kopfzeilen nach scr-Standard		
\ifx\chapter\undefined 		% falls Kapitel nicht definiert sind
  \automark[subsection]{section}% Kopf- und Fusszeilen setzen
\else				% Kapitel sind definiert
  \automark[section]{chapter}	% Kopf- und Fusszeilen setzen
\fi

%-----------------------------------------------------------------------------------
%   Maske f\"ur \"Uberschrift 
%-----------------------------------------------------------------------------------
% Belegung der notwendigen Kommandos f\"ur die Titelseite
\newcommand{\autor}{Klug, Nikolas} 		% bearbeitender Student
\newcommand{\veranstaltung}{Seminar zur Optimierung und Spieltheorie} 	% Titel des ganzen Seminars
\newcommand{\matrikelnummer}{1474569}
\newcommand{\uni}{Institut f\"ur Mathematik der Universit\"at Augsburg} % Universit\"at
\newcommand{\lehrstuhl}{Diskrete Mathematik, Optimierung und Operations Research} % Lehrstuhl
\newcommand{\semester}{Wintersemester 21/22}	% Winter- oder Sommersemester mit Jahr
\newcommand{\datum}{16.12.2021} 			% Datumsangabe
\newcommand{\thema}{The Ring Loading Problem}  		% Titel der Seminararbeit

\newcommand{\ownline}{\vspace{.7em}\hrule\vspace{.7em}} % horizontale Linie mit Abstand

\newcommand{\seminarkopf}{	% Befehl zum Erzeugen der Titelseite 
 \textsc{\autor}  \hfill{\datum} \\ 
\textbf{\veranstaltung} \\ 
\uni \\ 
\lehrstuhl \\
\semester \hfill{Matrikelnummer: \matrikelnummer}
\ownline 

\begin{center}
{\LARGE \textbf{\thema}}
\end{center}

\ownline
}			    % Befehle und Pakete f\"ur Titelseite
\input{header/theorem}			% Mathematische Befehle und Pakete

% Literatur-Bibliothek
%\bibliographystyle{alphadin}               % deutscher Bibliotheksstil

% Interaktive Referenzen, und PDF-Keys
\usepackage{xr-hyper}  
\usepackage[pagebackref,                
% R\"uckreferenz im Literaturverzeichnis
%           ps2pdf,  % Treiber f\"ur ps zu pdf ;           
           pdftex   % f\"ur direkt nach pdf
           ]{hyperref}

% Erweiterte Einstellungen zu hyperref

\hypersetup{
        breaklinks=true,        % zu lange Links unterbrechen
        colorlinks=true,        % F\"arben von Referenzen
        citecolor=black,        % Farbe der Zitate
        linkcolor=black,        % Farbe der Links
        extension=pdf,          % Externe Dokumente k\"onnen eingebunden werden.
        ngerman,		
	pdfview=FitH,
	pdfstartview=FitH,		
	bookmarksnumbered=true, % Anzeige der Abschnittsnummern	% pdf-Titel
	pdfauthor={\autor}% pdf-Autor
}

% Namen f\"ur Referenzen 

\newcommand{\ownautorefnames}{
  \renewcommand{\sectionautorefname}{Kapitel}
  \renewcommand{\subsectionautorefname}{Unterkapitel}
  \renewcommand{\subsubsectionautorefname}{\subsectionautorefname}
  \renewcommand{\appendixautorefname}{Anhang}
  \renewcommand{\figureautorefname}{Abbildung}
}

% R\"uckreferenzentext zur Literatur
\def\bibandname{und}%
\renewcommand*{\backref}[1]{}
\renewcommand*{\backrefalt}[4]{%
\ifcase #1 %
 (Nicht zitiert, also Erg\"anzungsliteratur.)%
\or
 (Cited on page #2.)%
\else
 (Cited on pages #2.)%
\fi
}
\renewcommand{\backreftwosep}{ and~} % seperate 2 pages
\renewcommand{\backreflastsep}{ and~} % seperate last of longer 

			% Befehle und Pakete f�r Referenzen
\usepackage{array}		% erweiterte Tabellen

% Schriftzeichen, Format
\usepackage{latexsym}		% Latex-Symbole
\usepackage[utf8]{inputenc}	% Eingabekodierungen
\usepackage[english]{babel}	% Mehrsprachenumgebung

% Layout
\usepackage{geometry}                    % Seitenränder
\geometry{a4paper, top=30mm, bottom=30mm, left=30mm, right=30mm}
\addtolength{\footskip}{-0.5cm}          % Seitenzahlen höher setzen
\usepackage{xcolor}                      % Farben


% Tabellen und Listen
\usepackage{float}		        % Gleitobjekte 
\usepackage[flushright]{paralist}       % Bessere Behandlung der Auflistungen

% Bilder
\usepackage[final]{graphicx}            % Graphiken einbinden

\usepackage{caption}                    % Beschriftungen
\usepackage{subcaption}                 % Beschriftungen f\"ur Unterteilung

%%%%%%
% Falls Zeichnungen mit pstricks erstellt werden sollen (Ausgabeprofil muss auf LaTeX -> PS -> PDF eingestellt werden)
%%%%%%
%\usepackage{pst-all}                    % Zeichnungen in Latex (kein pdflatex)
%\usepackage{pstricks-add}               % zus\"atzliches von pstricks
%\usepackage{pst-3dplot}                 % dreidimensionale Zeichnungen
%\usepackage{pst-eucl}                   % euklidisches Paket

\numberwithin{figure}{section}	% Abbildungsnummern in Section

%% My personal packages and config

\usepackage[super]{nth} % use superscripts for 1st, 2nd, 3rd
\usepackage{physics} % norm
\usepackage{bbm} % double struck numbers
\usepackage[]{algorithm2e}
\usepackage[activate=true,final,tracking=true,kerning=true,factor=1100,stretch=10,shrink=10]{microtype} % even better line spacing
\usepackage[capitalize,nameinlink,noabbrev]{cleveref} % better "\autoref"
\usepackage[colorinlistoftodos,prependcaption,textsize=tiny]{todonotes}
\usepackage{xargs} % Use more than one optional parameter in a new commands
\usepackage{enumitem} % changing enumeration styles
\usepackage[sort, numbers, square]{natbib} % citeauthor, citet
\usepackage{mathtools}
\usepackage{xspace} 
\usepackage[autostyle, english=american]{csquotes} % automatic left quotation marks
\usepackage{autonum} % auto equation numbering
\usepackage{mdframed}


\MakeOuterQuote{"}

% Captions for figures
\captionsetup{justification=raggedright, format=plain, font=small,labelfont=bf}

\newcommandx{\unsure}[2][1=]{\todo[linecolor=orange,backgroundcolor=orange!25,bordercolor=orange,#1]{#2}}
\newcommandx{\Todo}[2][1=]{\todo[linecolor=yellow,backgroundcolor=yellow!25,bordercolor=yellow,#1]{#2}}
\newcommandx{\info}[2][1=]{\todo[linecolor=green,backgroundcolor=green!25,bordercolor=green,#1]{#2}}

% Centered, equally spaced columns
\newcolumntype{Y}{>{\centering\arraybackslash}X} % centered equidistant columns

\setenumerate{label=(\arabic*),itemsep=0mm} 

\renewcommand{\epsilon}{\varepsilon}

\DeclareMathOperator*{\argmax}{arg\,max}
\DeclareMathOperator*{\argmin}{arg\,min}
\DeclareMathOperator{\JSD}{JS}
\DeclareMathOperator{\KL}{KL}
\newcommand{\T}{\mathrm{T}}
\newcommand{\R}{\mathbb{R}}
\newcommand{\cX}{\mathcal{X}}
\newcommand{\cG}{\mathcal{G}}
\newcommand{\cD}{\mathcal{D}}
\newcommand{\cF}{\mathcal{F}}
\newcommand{\cO}{\mathcal{O}}
\newcommand{\cZ}{\mathcal{Z}}
\newcommand{\cB}{\mathcal{B}}
\newcommand{\N}{\mathbb{N}}
\newcommand{\E}{\mathbb{E}}
\newcommand{\Z}{\mathbb{Z}}
\newcommand{\fL}{\mathfrak{L}}
\newcommand{\RL}{\textsc{RingLoading}\xspace}
\newcommand{\RRL}{\textsc{RRL}\xspace}
\newcommand{\RRLWC}{\textsc{RRLwC}\xspace}
\newcommand{\dOne}{\mathbbm{1}}
\newcommand{\Lopt}{L^{\mathrm{opt}}}

% boxed problem environment
\newenvironment{problem}[1]
{
%	\noindent\begin{fbox}
%		\begin{parbox}{\textwidth}
			\begin{center}
				\textsc{#1}
			\end{center}
}
{

%		\end{parsbox}
%	\end{fbox}
}
			    % restliche Befehle und Pakete

%%%%%%%%%%%%%%%%%%%%%%%%%%%%%%%%%%%%%%%%%%%%%%%%%%%%%%%%%%%%%%%%%%%%%%%%%%%%%%%%%%%%%%%%%%%
%%%%%%%%%%%%%%%%%%%%%%%%%%%%%%%%%%%%%%%%%%%%%%%%%%%%%%%%%%%%%%%%%%%%%%%%%%%%%%%%%%%%%%%%%%%
% Start des Dokuments
\begin{document}		

\selectlanguage{english}
%\ownautorefnames		% �nderung einiger automatischen Texte von hyperref (wie in referenz.tex definiert)
\parindent0em 			% kein Einzug nach einer Leerzeile

%%%%%%%%%%%%%%%%%%%%%%%%%%%%%%%%%%%%%%%%%%%%%%%%%%%%%%%%%%%%%%%%%%%%%%%%%%%%%%%%%%%%%%%%%%%
% Titelseite
\thispagestyle{empty}		% leerer Seitenstil, also keine Seitennummer auf der Titelseite
\begin{titlepage}
\seminarkopf 			% Titelblatt (wie in kopf.tex definiert)
\begin{abstract}
	The ring loading problem arises during the planning of SONET rings.
	Given a cycle graph of $n > 1$ nodes and a traffic demand for each pair of nodes, the problem is to determine a routing in which each demand is routed either completely in the clockwise or completely in the counterclockwise direction around the ring, while minimizing the maximal load on any edge.
	
	Since this problem is NP-complete, Schrijver, Seymour and Winkler presented an approximation algorithm in their paper "The Ring Loading Problem".
	This algorithm computes approximations that exceeds the optimal solution by at most $\frac{3}{2}D$, where $D$ is the maximal demand.
	In the process, an optimal solution to the relaxed ring loading problem is determined, which allows demands to be routed both ways at the same time.
	
	In this work, Schrijver et al.'s algorithm is presented and its is correctness proven.	
\end{abstract} 
\end{titlepage}

%%%%%%%%%%%%%%%%%%%%%%%%%%%%%%%%%%%%%%%%%%%%%%%%%%%%%%%%%%%%%%%%%%%%%%%%%%%%%%%%%%%%%%%%%%%
% Inhaltsverzeichnis
\thispagestyle{empty}	
\tableofcontents		% Inhaltsverzeichnis
%\listoffigures			% Abbildungsverzeichnis (eventuell einf�gen)
%\listoftables			% Tabellenverzeichnis (eventuell einf�gen)
\setcounter{page}{0}% Eigentlicher Inhalt beginnt auf Seite 1
\clearpage          % neue Seite f�r eigentlichen Inhalt
%%%%%%%%%%%%%%%%%%%%%%%%%%%%%%%%%%%%%%%%%%%%%%%%%%%%%%%%%%%%%%%%%%%%%%%%%%%%%%%%%%%%%%%%%%%
% Eigentlicher Inhalt der Seminararbeit; die einzelnen Teile werden hier (aus Gr�nden der �bersichtlichkeit) �ber \input{file} eingebunden

\section{Introduction}

In this work, the 1999 paper "The Ring Loading Problem" by Schrijver and Seymour will be presented and discussed.
This will be the introduction.

This work is based on \citet{schrijver99}.

Already shown: Tight bound $L^\mathrm{opt} \leq 2 L^\ast$ (take references from däubel intro!! TODO).

\citet{schrijver99} achieve an additive bound of $L^\mathrm{opt} + \frac{3}{2}D$.
This bound has since been improved to $1.3$ but the improvement based on the same algorithm.
It is therefore still sensible to review and understand the algorithm presented by \citet{schrijver99} in the late 1990s.




%\section{Ring Graphs and Terminology}
%
%\section{Ring Loading Problems and Complexity}

Definition of (Integer) Ring Loading also as decision problem

Definition of Relaxed Ring Loading also as decision problem

Complexity of Ring Loading and Relaxed Ring Loading

Interval-notation for ring interpreted as equivalence classes of finite ring.

\begin{notation}
	Let $1 \leq i < j \leq n$.
	Then we write $[i, j) \coloneqq \{i, i+1, \ldots, j-2, j-1\}$.
	We also define $[j, i) \coloneqq \{j, j+1, \ldots, n-1, n, 1, \ldots, i-2, i-1\}$.
	Open, half-open and closed intervals are to be interpreted in the same manner as real intervals.
\end{notation}

\begin{problem}{Ring Loading}
	\textbf{In:} Ring of size $n \in \N$, demands $d_{ij} \in \R^+$ for $1 \leq i < j \leq n$.\\
	\textbf{Goal:} Find a function $\Phi: \{(i, j)~|~1 \leq i < j \leq n\} \rightarrow \{0, 1\}$ such that
	$\max_{1 \leq i \leq n} L_i$ is minimal, where
	\begin{equation}
		L_i(\phi) \coloneqq \ldots
	\end{equation}
\end{problem}
Ring loading is a routing problem where traffic must be routed either through the front or the back.

The cases $n = 1, 2$ are trivial and $n = 3$ has a solution independent of demands (see figure XX). Hence we assume $n \geq 4$ at all times.

$\phi(i, j) = 1$ means that the traffic from the demand $d_{ij}$ is routed through the path $\{i, i+1, \ldots, j-1, j\}$.
This route is called the \emph{front route}.
Otherwise, if $\phi(i, j) = 0$, the traffic is routed through $\{j, j+1, \ldots, n-1, n, 1, \ldots, i\}$, which is called the \emph{back route}.
The back route always contains the link $\{n, 1\}$.

\begin{theorem}
	\RL in its decision form is NP-complete.
\end{theorem}
\begin{proof}
	\Todo{Input can be encoded in O(n**2) space}
	A function $\Phi$ of the desired form can be encoded using $\cO(n^2)$ space, e.g. as a binary array, and serves as witness for \RL.
	This implies that \RL is in NP.
	In order to show NP-completeness, we provide a polynomial-reduction of the \textsc{Partition} problem \cite{karp72}.
	This problem asks whether, given $m$ integers $\{z_1, \ldots, z_m\}$, there exists an $S \subseteq [m]$ such that 
	\begin{equation}
		\sum_{i \in S} z_i = \sum_{j \in [m] \setminus S} z_j \ .
	\end{equation}
	
\end{proof}

\begin{definition}
	Let $1 \leq g < h \leq n$.
	A \emph{cut} $\{g, h\}$ is a set of two links $\{\{g, g+1\}, \{h, h+1\}\}$.
	A demand $d_{ij}$ is said to \emph{cross} a cut $\{g, h\}$ if exactly one of $i$ and $j$ lies within $[g, h)$.
	The \emph{total demand across $\{g, h\}$} is defined as
	\begin{equation}
		D_{gh} \coloneqq \sum_{\{d_{ij}~|~d_{ij}\ \text{crosses}\ \{g, h\}\}} d_{ij} \ .
	\end{equation}
	This is the sum of the demands that must cross either $\{g, g+1\}$ or $\{h, h+1\}$.
\end{definition}

\begin{figure}
	\caption{Ring, demands, cuts and more }
	\label{fig:cut-example}
\end{figure}

We can formulate a relaxed version of \RL, which allows demands to be routed both ways around the ring.
\begin{problem}{Relaxed Ring Loading}
	\textbf{In:} Ring of size $n \in \N$, demands $d_{ij} \in \R^+$ for $1 \leq i < j \leq n$.\\
	\textbf{Goal:} Find a function $\phi: \{(i, j)~|~1 \leq i < j \leq n\} \rightarrow [0, 1]$ such that $\max_{1 \leq i \leq n} L_i^\ast$ is minimal, where
	\begin{equation}
		L_i^\ast(\phi) \coloneqq \ldots
	\end{equation}
\end{problem}
In contrast to its binary version, \RRL can be solved in polynomial time, as it can be formulated as the following linear problem
\begin{alignat}{2}
	&\min &\quad& L\\
	&s.t. &\quad& L \geq L_i^\ast(\Phi)\quad \forall 1 \leq i \leq n \ .
\end{alignat}
Note that the $L_i$ are linear functions.


\section{An Algorithm for Relaxed Ring Loading}

\section{An Algorithm for Ring Loading}
\label{sec:ring-loading}

In this section, we see how we can transform the real-valued routing $\Phi$ determined by \cref{algo:rrl} into a binary routing that approximates the optimal routing.
Let $L_i = L_i(\Phi)$ be the edges loads induced by $\Phi$ and $S$ be the set of demands that are split.
In order to obtain a binary routing, we simply have to reroute the demands in $S$ all front or all back -- ideally without increasing the maximal edge load by too much. 
As we have seen, all demands in $S$ are mutually crossing and $\abs{S} \leq \lfloor \frac{n}{2} \rfloor$.
That means, each node is an endpoint of at most one demand.
For simplicity, we remove all nodes $i \in [n]$ that are not endpoints of any demand.
When rerouting any split demand, the loads on the two edges incident to $i$ change by the same amount.
Hence, we replace the edges $\{i-1, i\}$ and $\{i, i+1\}$ with a single new edge $\{i', i'+1\}$ and set 
\begin{equation}
	L_{i'} \coloneqq \max(L_{i-1}, L_i) \ .
\end{equation}
For rerouting the split demands, it suffices to consider this contracted instance $I'$ of size $2m$, where $m \coloneqq \abs{S}$.
We can rewrite $S$ in a simpler way as $S' = \{d_{i, i+m}\ |\ i \in [m]\}$.

For all $d_{i, i+m} \in S'$, let $x_i$, $y_i$ be the amount of traffic of $d_{i, i+m}$ that $\Phi$ routes through the front and the back, respectively.
Rerouting $d_{i, i+m}$, e.g. by routing it all front increases the the edge loads on the front route of $d_{i, i+m}$ by $y_i$, while those on the back route are decreased by $y_i$.
Similarly, routing $d_{i, i+m}$ all back increases the loads on the back route and decreases those in the front by $x_i$, respectively.

Now, assume that all demands in $S'$ have been routed either all front or all back.
For each $d_{i, i+m} \in S'$ we define
\begin{equation}
	z_i = \begin{cases}
		y_i, &\text{if } d_{i, i+m} \text{ is routed all front} \\
		-x_i, &\text{if } d_{i, i+m} \text{ is routed all back}
	\end{cases} \ ,
\end{equation}
that is, $z_i$ is the change of traffic on front route of $d_{i, j}$.
This allows us to formulate the total change of traffic on any edge $\{k, k+1\}$, $k \in [2m]$ and determine the new link load $L_k'$:
\begin{equation}
	\label{eq:load-change}
	L_k' = L_k + \sum_{i \in [m], k \in [i, i+m)} z_i - \sum_{i \in [m], k \notin [i, i+m)} z_i \ .
\end{equation}
Here, we have $k \in [i, i+m)$ if and only if the edge $\{k,k+1\}$ lies on the front route of $d_{i, i+m}$.


The following theorem shows that there is a way to reroute the demands in $S'$ such that the maximal edge load remains bounded.
\begin{lemma}
	\label{lemma:reroute-demands}
	Let $x_i, y_i \in \R_{\geq 0}, i \in [m]$ be such that $x_i + y_i \leq D$ for $D \in \R_{\geq 0}$.
	Then there exist $z_i \in \R$, $i \in [m]$ such that:
	\begin{align}
		(z_i = y_i \quad \text{ or } \quad z_i = -x_i) 
		\quad \text{ and } \quad \sum_{i=1}^k z_i \in \left[-\frac{D}{2}, \frac{D}{2}\right] \quad \forall k \in [m] \ .
	\end{align}
\end{lemma}
\begin{proof}
	We iteratively define the $z_i$.
	Set $z_1 \coloneqq y_1$.
	Then, for $k \in [n]$, let $z_i, 1 \leq i < k$ be such that $\sum_{i=1}^{k-1} z_i \in \left[-\frac{D}{2}, \frac{D}{2}\right]$.
	Since $x_k + y_k \leq D$, at least one of the inequalities $\sum_{i=1}^{k-1} z_i + y_k \leq \frac{D}{2}$, $\sum_{i=1}^{k-1} z_i - x_i \geq -\frac{D}{2}$ holds true.
	Choose $z_k = y_k$ or $z_k = -x_k$ accordingly.	
\end{proof}
\begin{theorem}
	\label{theo:ring-loading-algorithm}
	Let $\Phi$ be the routing obtained from \cref{algo:rrl} and $\Lopt$ be the ringload of an optimal binary routing.
	Then the demands that are split by $\Phi$ can be rerouted such that the resulting binary routing $\Phi'$ has a ringload of
	\begin{equation}
		L(\Phi') \leq \Lopt + \frac{3}{2}D \ ,
	\end{equation}
	where $D = \max_{1 \leq i < j \leq n} d_{i, j}$ is the largest demand.
\end{theorem}
\begin{proof}
	For all $d_{i, i+m} \in S'$, let $x_i$, $y_i$ be the amount of traffic of $d_{i, i+m}$ that $\Phi$ routes through the front and the back, respectively.
	Choose $z_i, i \in [m]$ like in \cref{lemma:reroute-demands} with $D = \max_{1 \leq i < j \leq n} d_{i, j}$.
	We construct the binary routing $\Phi'$ by setting $\Phi'(i, j) \coloneqq \Phi(i, j)$ for all $d_{i,j} \notin S$.
	For all demands $d_{i, j}$ that are split by $\Phi$, we set
	\begin{equation}
		\Phi'(i, j) = \begin{cases}
			1, & \text{if } z_i = y_i \\
			0, & \text{if } z_i = -x_i
		\end{cases}
	\end{equation}
	where $z_i$ corresponds to the demand $d_{i, j}$ in $S$.
	We get:
	\begin{align}
		L(\Phi') - L(\Phi) &= \max_{k \in [n]} (L_k(\Phi') - L(\Phi)) 
		\leq \max_{k \in [n]} (L_k(\Phi') - L_k(\Phi))\\
		& \stackrel{\ref{eq:load-change}}{=} \max_{k \in [m]} \left(\sum_{i \in [m], k \in [i, i+m)} z_i - \sum_{i \in [m], k \notin [i, i+m)} z_i\right)\\
		&= \max_{k \in [m]} \abs{\sum_{i = 1}^k z_i - \sum_{i = k+1}^m z_i }\\
		&= \max_{k \in [m]} \abs{2\sum_{i = 1}^k z_i - \sum_{i = 1}^m z_i } \ . \label{eq:partial-reroutes}
	\end{align}
	Because both $\sum_{i = 1}^k z_i$ and $\sum_{i = 1}^m z_i$ lie in the interval $[-\frac{D}{2}, \frac{D}{2}]$, 
	we have 
	\begin{equation}
		2\sum_{i = 1}^k z_i - \sum_{i = 1}^m z_i \in [-\frac{3}{2}D, \frac{3}{2}D] \ .
	\end{equation}
	Thus we get $L(\Phi') - L(\Phi) \leq \frac{3}{2}D$, or equivalently $L(\Phi') \leq L(\Phi) + \frac{3}{2}D$.
	Observing that $L(\Phi) \leq \Lopt$ concludes the proof.
\end{proof}

%\section{Runtime Improvements}
\label{sec:runtime}

In this section I show that with some modifications, Schrijver et al.'s \cite{schrijver99} algorithm for \RRL (and therefore also \RL) can be improved to $\cO(n^2)$.

We do this by examining the steps of \cref{algo:rrl}.
In Step 1, the demands across all cuts are computed.
Doing this in the naïve way -- by using the definition in \cref{eq:cut-demand-definition} takes $O(n^2)$ time per cut, which means $O(n^4)$ time in total.
A more efficient way can be derived from the following lemma.
\begin{lemma}
	The following recursion holds true for all $g, h \in [n]$, $g < h$:
	\begin{equation}
		\label{eq:recursive-dac}
		D_{g, h} = \begin{cases}
			\sum_{k < g+1} d_{k,g+1} + \sum_{k > g+1} d_{g+1, k}, & \text{if } h = g+1 \\
			D_{g, g+1} + D_{g+1, h} - 2 A_{g+1, h}, & \text{if } h > g+1
		\end{cases} \ ,
	\end{equation}
	where the auxiliary variables $A_{g, h}$ are defined as:
	\begin{equation}
		A_{g, h} \coloneqq \begin{cases}
			d_{g, g+1}, & \text{if } h = g+1\\
			d_{g, h} + A_{g, h-1}, &\text{if } h > g+1\\
		\end{cases} \ .
	\end{equation}
\end{lemma}
%We note that \cref{eq:recursive-dac} is in fact a recursion, since in the case of $h > g+1$, only demands across "smaller" cuts are used.
\begin{proof}
	Let $\{g, h\}$, $g < h$ be a cut.
	If $h = g+1$, the demand across $\{g, h\}$ is exactly the sum of all demands with one of their endpoints equal to $g+1$.
	
	Now, let $h > g+1$ and let $B$ be the set of all demands with exactly one endpoint in $(g, h]$, meaning that $D_{g, h}$ is the sum over $B$.	
	In the following, let w.l.o.g $i \in (g, h]$.
	We rewrite $B$ as: 
	\begin{align}
		B &= \{d_{i, j}\ |\ \abs{\{i, j\} \cap (g, h]} = 1\} \\
 		&=\{d_{i, j}\ |\ i = g+1 \text{ and } j \notin(g, h]\} \cup \{d_{i, j}\ |\ i \in (g+1, h] \text{ and } j \notin(g, h]\} \\
		&= (\underbrace{\{d_{i, j}\ |\ \abs{\{i, j\} \cap (g, g+1]} = 1\}}_{\eqqcolon B_1} \setminus \underbrace{\{d_{i, j}\ |\ i = g+1 \text{ and } j \in (g+1, h]\}}_{\eqqcolon A})\\
		& \cup
			(\underbrace{\{d_{k, i}\ |\ \abs{\{k, i\} \cap (g+1, h]} = 1 \}}_{\eqqcolon B_2} \setminus \underbrace{\{d_{k, i}\ |\ k = g+1 \in  \text{ and } i \in (g+1, h]\}}_{= A}) \ .
	\end{align}
	Now, the sum over $B_1$ and $B_2$ is $D_{g, g+1}$ and $D_{g+1, h}$, respectively.
	The sum over $A$ is $\sum_{a = g+1}^h d_{g, a}$.
	Hence $D_{g, h} = D_{g, g+1} + D_{g+1, h} - 2 \sum_{a = g+1}^h d_{g, a}$.
	One can verify that $A_{g+1, h} = \sum_{a = g+1}^h d_{g, a}$, concluding the proof.
\end{proof}
The recursion in \cref{eq:recursive-dac} can be implemented using $\cO(n^2)$ time.
This is quite remarkable, as it means that in $\cO(n^2)$ time, we can either compute the demand across one cut, or the demands across all cuts.

Using \cref{eq:minimal-capacities}, Step 2 runs in $\cO(n^2)$.

In Step 3, all but at most $\lfloor \frac{n}{2} \rfloor$ demands are routed all front or all back.
The result of \cref{lemma:tight-cuts-remain-tight} is that we can find a tight cut for each edge in the beginning and reuse those, instead of searching for new tight cuts in each iteration.
Effectively, this means that in Step 3, all demands that are parallel to one of these initial tight cuts are routed all front or all back -- or at least could be without harm due to \cref{lemma:parallel-to-tight-cut}.
This observation also alleviates adjusting capacities and recomputing the demands across cuts in each iteration.
Altogether, we can replace Step 3 with the following subroutine:

\begin{algorithm}[H]
	\begin{mdframed}[backgroundcolor=green!10,linecolor=white,innerleftmargin=25pt,leftmargin=-25pt,rightmargin=15pt]
		$t[g] = h$, where $h \in [n]$ such that $\{g, h\}$ is a tight cut, for all $g \in [n]$\;
		$r[i] = i+1$ for all $i \in [n]$\;
		\For{$g \in [n]$}{
			\For{$i \in [n]$}{
				\While{$r[i] \neq i$ and $d_{i, r[i]}$ is parallel to $\{g, t[g]\}$}{
					 Route $d_{i, r[i]}$ to miss $\{g, h\}$\;
					 $r[i] = ((r[i] + 1) \mod n) + 1$\;
				}	
			}
		}
	\end{mdframed}
	\caption{A $\cO(n^2)$ subroutine replacing Step 3 of \cref{algo:rrl}.}
	\label{algo:step3}
\end{algorithm}
Here, the idea is that at all times, if $r[i] > j$, the demand $d_{i,j}$ (or $d_{j, i}$) has already been routed.
It can be shown by contraposition that if a demand $d_{i,j}$ is parallel to any of the initial tight cuts $\{g, h\}$, it is routed before or in the $g$-th iteration of the outer for-loop.
Also, one can show that all demands that are not parallel to any of the initial tight cuts are mutually crossing.
Hence the new subroutine computes an output with the same properties as Step~3 in \cref{algo:rrl}.
We furthermore observe that for each $i \in [n]$, $r[i]$ is increased at most $n$ times, implying the runtime is in $\cO(n^2)$.

While the edge capacities must not be updated in each iteration of Step~3, they must be known before Step~4.
The residual capacities are simply the difference between the original capacities and the edge loads.
Again using a recursion, the edge loads resulting from the binary "pre-routing" obtained in Step~3 can be computed in $\cO(n^2)$ time.

\begin{lemma}
	\label{lemma:link-loads}
	For all $k \in [n]$ let $L_k^\mathrm{f}(\Phi)$ and $L_k^\mathrm{b}(\Phi)$ be the total traffic that is routed forward and backward, respectively, through the edge $\{k, k+1\}$ by the routing $\Phi$. 
	With $L_0^\mathrm{b} \coloneqq L_n^\mathrm{b}$, the following holds true:
		\begin{align}
		L_k^\mathrm{f}(\Phi) &\coloneqq \begin{cases}
			\sum_{j = 2}^n d_{1,j} \cdot \Phi(1, j), & \text{if } i = 1 \\
			L_{i-1}^\mathrm{f} - \sum_{j = 1}^{i-1} d_{j,i} \cdot \Phi(j, i) + \sum_{j = i}^{n} d_{i,j} \cdot \Phi(i, j) & \text{if } i > 1
		\end{cases} \ , \\
		L_k^\mathrm{b}(\Phi) &\coloneqq \begin{cases}
			\sum_{i = 1}^n \sum_{j = i+1}^n d_{i,j} \cdot (1 - \Phi(i, j)) & \text{if } i = n\\
			L_{i-1}^\mathrm{b} - \sum_{j = i}^{n} d_{i, j} \cdot (1 - \Phi(i, j)) + \sum_{j = 1}^{i-1} d_{i,j} \cdot(1 - \Phi(i, j)) & \text{if } i < n
		\end{cases} \ .
	\end{align}
\end{lemma}

In Step~4 of \cref{algo:rrl}, the minimal slacks on the front routes of the remaining demands have to be determined.
Doing this the naïve way takes $\cO(n^2)$ time per demand, so $\cO(n^3)$ time in total (again utilizing \cref{eq:recursive-dac}).
This, too, can be improved to $\cO(n^2)$.
The precise way to do this is quite lengthy and can hence not be discussed in this work. 
The interested reader is instead referred to the appendix.

%\section{The Constant}

The constant $\frac{3}{2}$ in \cref{theo:ring-loading-algorithm}

\section{Conclusion}

In this work, the approximation algorithm for \RL by \citet{schrijver99} was presented.
On the way, we have seen how an, exact solution to \RRL can be determined in $\cO(k n^2)$ time, where $k$ is the number of non-zero demands.
In this solution, all but at most $\lfloor \frac{n}{2} \rfloor$ demands are split.
These are rerouted in a way to achieve a ringload of $L \leq \Lopt + \frac{3}{2}D$, where $\Lopt$ is the optimal binary ringload and $D$ the maximal demand.
Since its publication, the algorithm has further been used by \citet{khanna97} to obtain a polynomial-time approximation scheme for \RL.

Also, there is still active research on the approximation quality of Schrijver et al.'s algorithm:
It turns out that the constant $\frac{3}{2}$ is only a rough upper bound.
The "best" upper bound $\alpha \in \R$ would be the infimum of all $\beta \in \R$ such that the following holds true:
\begin{quote}
	For all $n \in \N$, $x_i, y_i \in \R_{\geq 0}$ such that $x_i + y_i \leq 1$ for all $i \in [n]$, there exist $z_i \in \R_{\geq 0}$ such that for all $i \in [n]$
	\begin{equation}
		(z_i = y_i \quad \text{ or } \quad z_i = -x_i) 
		\quad \text{ and } \quad \abs{\sum_{i=1}^k z_i - \sum_{i = 1}^n z_i} \leq \beta \quad \forall k \in [n] \ .
	\end{equation}
\end{quote}
The latest results on this problem have shown that $\alpha \leq \frac{19}{14}$ and $\alpha \geq \frac{11}{10}$ \cite{skutella16, daubel19}.


To this day, the exact upper bound remains an open question.

\clearpage          % neue Seite f�r Literaturverzeichnis

%%%%%%%%%%%%%%%%%%%%%%%%%%%%%%%%%%%%%%%%%%%%%%%%%%%%%%%%%%%%%%%%%%%%%%%%%%%%%%%%%%%%%%%%%%%%
% Literaturverzeichnis
%\nocite*  % Nicht zitierte Quellen werden auch ins Literaturverzeichnis aufgenommen
\thispagestyle{empty}
\bibliography{bibliography}
\bibliographystyle{plainnat}  % Literaturverzeichnis liegt in der Datei seminararbeit.bib

%%%%%%%%%%%%%%%%%%%%%%%%%%%%%%%%%%%%%%%%%%%%%%%%%%%%%%%%%%%%%%%%%%%%%%%%%%%%%%%%%%%%%%%%%%%%
%%%%%%%%%%%%%%%%%%%%%%%%%%%%%%%%%%%%%%%%%%%%%%%%%%%%%%%%%%%%%%%%%%%%%%%%%%%%%%%%%%%%%%%%%%%%
% Ende des Dokuments
\end{document}			
