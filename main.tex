\documentclass[paper=a4, 	% Seitenformat
		fontsize=11pt, 		% Schriftgr\"o\ss{}e
		abstract=true, 	% mit Abstrakt
		headsepline, 	% Trennlinie f\"ur die Kopfzeile
		notitlepage	% keine extra Titelseite
		]{scrartcl}

%%%%%%%%%%%%%%%%%%%%%%%%%%%%%%%%%%%%%%%%%%%%%%%%%%%%%%%%%%%%%%%%%%%%%%%%%%%%%%%%%%%%%%%%%%%
% Zusammenfassung einiger n�tzlicher Pakete und Befehle
%-----------------------------------------------------------------------------------
% Kopf-Zeilen
%-----------------------------------------------------------------------------------

\usepackage[automark]{scrlayer-scrpage}	% Seiten-Stil f\"ur scrartcl
\pagestyle{scrheadings}		% Kopfzeilen nach scr-Standard		
\ifx\chapter\undefined 		% falls Kapitel nicht definiert sind
  \automark[subsection]{section}% Kopf- und Fusszeilen setzen
\else				% Kapitel sind definiert
  \automark[section]{chapter}	% Kopf- und Fusszeilen setzen
\fi

%-----------------------------------------------------------------------------------
%   Maske f\"ur \"Uberschrift 
%-----------------------------------------------------------------------------------
% Belegung der notwendigen Kommandos f\"ur die Titelseite
\newcommand{\autor}{Klug, Nikolas} 		% bearbeitender Student
\newcommand{\veranstaltung}{Seminar zur Optimierung und Spieltheorie} 	% Titel des ganzen Seminars
\newcommand{\matrikelnummer}{1474569}
\newcommand{\uni}{Institut f\"ur Mathematik der Universit\"at Augsburg} % Universit\"at
\newcommand{\lehrstuhl}{Diskrete Mathematik, Optimierung und Operations Research} % Lehrstuhl
\newcommand{\semester}{Wintersemester 21/22}	% Winter- oder Sommersemester mit Jahr
\newcommand{\datum}{16.12.2021} 			% Datumsangabe
\newcommand{\thema}{The Ring Loading Problem}  		% Titel der Seminararbeit

\newcommand{\ownline}{\vspace{.7em}\hrule\vspace{.7em}} % horizontale Linie mit Abstand

\newcommand{\seminarkopf}{	% Befehl zum Erzeugen der Titelseite 
 \textsc{\autor}  \hfill{\datum} \\ 
\textbf{\veranstaltung} \\ 
\uni \\ 
\lehrstuhl \\
\semester \hfill{Matrikelnummer: \matrikelnummer}
\ownline 

\begin{center}
{\LARGE \textbf{\thema}}
\end{center}

\ownline
}			    % Befehle und Pakete f\"ur Titelseite
% Mathematische Zeichens\"atze und Umgebungen
\usepackage{amsfonts, amssymb}	% Definition einer Liste mathematischer Fontbefehle und Symbole
\usepackage[intlimits,		% Integralgrenzen \"uber und unter dem Integral
	    sumlimits]		% Summationsgrenzen \"uber und unter der Summe
           {amsmath}		% mathematische Verbesserungen
\usepackage{amsthm}		% spezielle theorem Stile
\usepackage{aliascnt} 

%-----------------------------------------------------------------------------------
% Hilfreiche Befehle
%-----------------------------------------------------------------------------------
\newcommand{\betrag}[1]{\lvert #1 \rvert}	        % Betrag
\newcommand{\norm}[1]{\lVert #1 \rVert}		        % Norm
\providecommand*{\Lfloor}{\left\lfloor}                 % gro\ss{}es Abrunden
\providecommand*{\Rfloor}{\right\rfloor}                % gro\ss{}es Abrunden
\providecommand*{\Floor}[1]{\Lfloor #1 \Rfloor}         % gro\ss{}es ganzes Abrunden
\providecommand*{\Ceil}[1]{\left\lceil #1 \right\rceil} % gro\ss{}es ganzes Aufrunden

\DeclareMathOperator{\e}{ex}
\DeclareMathOperator{\ma}{mate}
\DeclareMathOperator{\Ex}{Ex}

%-----------------------------------------------------------------------------------
%   Befehle f\"ur Nummerierung der Ergebnisse
%   fortlaufend innerhalb eines Abschnittes
%-----------------------------------------------------------------------------------
\theoremstyle{plain}            % normaler Stil
\newtheorem{theorem}{Theorem}[section]
% Lemma
\newaliascnt{lemma}{theorem}  
\newtheorem{lemma}[lemma]{Lemma}  
\aliascntresetthe{lemma}  
% Satz
\newaliascnt{satz}{theorem}  
\newtheorem{satz}[satz]{Satz} 
\aliascntresetthe{satz}
% Korollar
\newaliascnt{korollar}{theorem}  
\newtheorem{korollar}[korollar]{Korollar} 
\aliascntresetthe{korollar}
% Proposition
\newaliascnt{proposition}{theorem}  
\newtheorem{proposition}[proposition]{Proposition} 
\aliascntresetthe{proposition}
% Invariant
\newaliascnt{invariant}{theorem}
\newtheorem{invariant}[invariant]{Invariant}
\aliascntresetthe{invariant}
%-----------------------------------------------------------------------------------
\theoremstyle{definition}	% Definitionsstil
% Definition
\newaliascnt{definition}{theorem}  
\newtheorem{definition}[definition]{Definition} 
\aliascntresetthe{definition}
% Beispiel
\newaliascnt{beispiel} {theorem}  
\newtheorem{beispiel}[beispiel]{Beispiel} 
\aliascntresetthe{beispiel} 
% Problem
%\newaliascnt{problem}{theorem}  
%\newtheorem{problem}[problem]{Problem} 
%\aliascntresetthe{problem}
% Algorithmus
\newaliascnt{algorithmus}{theorem}  
\newtheorem{algorithmus}[algorithmus]{Algorithmus} 
\aliascntresetthe{algorithmus} 
%-----------------------------------------------------------------------------------
\theoremstyle{remark}		% Bemerkungsstil
% Bemerkung
\newaliascnt{bemerkung}{theorem}  
\newtheorem{bemerkung}[bemerkung]{Bemerkung}  
\aliascntresetthe{bemerkung} 
% Vermutung
\newaliascnt{vermutung}{theorem}  
\newtheorem{vermutung}[vermutung]{Vermutung}  
\aliascntresetthe{vermutung} 
% Notation
\newaliascnt{notation}{theorem}  
\newtheorem{notation}[notation]{Notation} 
\aliascntresetthe{notation}

%-----------------------------------------------------------------------------------
% automatische Referenzen mit interaktivem Text
%-----------------------------------------------------------------------------------

% Texte
\newcommand{\theoremautorefname}{Theorem}
\newcommand{\lemmaautorefname}{Lemma}
\newcommand{\satzautorefname}{Satz}
\newcommand{\korollarautorefname}{Korollar}
\newcommand{\propositionautorefname}{Proposition}
\newcommand{\invariantautorefname}{Invariant}

\newcommand{\definitionautorefname}{Definition}
\newcommand{\beispielautorefname}{Beispiel}
%\newcommand{\problemautorefname}{Problem}
\newcommand{\algorithmusautorefname}{Algorithmus}

\newcommand{\bemerkungautorefname}{Bemerkung}
\newcommand{\vermutungautorefname}{Vermutung}
\newcommand{\notationautorefname}{Notation}

%-----------------------------------------------------------------------------------
% Nummerierung der Gleichungen innerhalb der obersten Ebene
%-----------------------------------------------------------------------------------
\ifx\chapter\undefined 			% Kapitel sind definiert
  \numberwithin{equation}{section}	% Gleichungsnummern in Section
\else					% Kapitel sind nicht definiert
  \numberwithin{equation}{chapter}	% Gleichungsnummern in Kapiteln
\fi
			% Mathematische Befehle und Pakete

% Literatur-Bibliothek
%\bibliographystyle{alphadin}               % deutscher Bibliotheksstil

% Interaktive Referenzen, und PDF-Keys
\usepackage{xr-hyper}  
\usepackage[pagebackref,                
% R\"uckreferenz im Literaturverzeichnis
%           ps2pdf,  % Treiber f\"ur ps zu pdf ;           
           pdftex   % f\"ur direkt nach pdf
           ]{hyperref}

% Erweiterte Einstellungen zu hyperref

\hypersetup{
        breaklinks=true,        % zu lange Links unterbrechen
        colorlinks=true,        % F\"arben von Referenzen
        citecolor=black,        % Farbe der Zitate
        linkcolor=black,        % Farbe der Links
        extension=pdf,          % Externe Dokumente k\"onnen eingebunden werden.
        ngerman,		
	pdfview=FitH,
	pdfstartview=FitH,		
	bookmarksnumbered=true, % Anzeige der Abschnittsnummern	% pdf-Titel
	pdfauthor={\autor}% pdf-Autor
}

% Namen f\"ur Referenzen 

\newcommand{\ownautorefnames}{
  \renewcommand{\sectionautorefname}{Kapitel}
  \renewcommand{\subsectionautorefname}{Unterkapitel}
  \renewcommand{\subsubsectionautorefname}{\subsectionautorefname}
  \renewcommand{\appendixautorefname}{Anhang}
  \renewcommand{\figureautorefname}{Abbildung}
}

% R\"uckreferenzentext zur Literatur
\def\bibandname{und}%
\renewcommand*{\backref}[1]{}
\renewcommand*{\backrefalt}[4]{%
\ifcase #1 %
 (Nicht zitiert, also Erg\"anzungsliteratur.)%
\or
 (Zitiert auf Seite #2.)%
\else
 (Zitiert auf den Seiten #2.)%
\fi
}
\renewcommand{\backreftwosep}{ und~} % seperate 2 pages
\renewcommand{\backreflastsep}{ und~} % seperate last of longer 

			% Befehle und Pakete f�r Referenzen
\usepackage{array}		% erweiterte Tabellen

% Schriftzeichen, Format
\usepackage{latexsym}		% Latex-Symbole
\usepackage[latin1]{inputenc}	% Eingabekodierungen
\usepackage[english]{babel}	% Mehrsprachenumgebung

% Layout
\usepackage{geometry}                    % Seitenränder
\geometry{a4paper, top=30mm, bottom=30mm, left=30mm, right=30mm}
\addtolength{\footskip}{-0.5cm}          % Seitenzahlen höher setzen
\usepackage{xcolor}                      % Farben


% Tabellen und Listen
\usepackage{float}		        % Gleitobjekte 
\usepackage[flushright]{paralist}       % Bessere Behandlung der Auflistungen

% Bilder
\usepackage[final]{graphicx}            % Graphiken einbinden

\usepackage{caption}                    % Beschriftungen
\usepackage{subcaption}                 % Beschriftungen f\"ur Unterteilung

%%%%%%
% Falls Zeichnungen mit pstricks erstellt werden sollen (Ausgabeprofil muss auf LaTeX -> PS -> PDF eingestellt werden)
%%%%%%
%\usepackage{pst-all}                    % Zeichnungen in Latex (kein pdflatex)
%\usepackage{pstricks-add}               % zus\"atzliches von pstricks
%\usepackage{pst-3dplot}                 % dreidimensionale Zeichnungen
%\usepackage{pst-eucl}                   % euklidisches Paket

\numberwithin{figure}{section}	% Abbildungsnummern in Section

%% My personal packages and config

\usepackage[super]{nth} % use superscripts for 1st, 2nd, 3rd
\usepackage{physics} % norm
\usepackage{bbm} % double struck numbers
\usepackage[ruled,noline]{algorithm2e}
\usepackage[activate=true,final,tracking=true,kerning=true,factor=1100,stretch=10,shrink=10]{microtype} % even better line spacing
\usepackage[capitalize,nameinlink,noabbrev]{cleveref} % better "\autoref"
\usepackage[colorinlistoftodos,prependcaption,textsize=tiny]{todonotes}
\usepackage{xargs} % Use more than one optional parameter in a new commands
\usepackage{enumitem} % changing enumeration styles
\usepackage[sort, numbers, square]{natbib} % citeauthor, citet
\usepackage{mathtools}
\usepackage{xspace} 
\usepackage[autostyle, english=american]{csquotes} % automatic left quotation marks


\MakeOuterQuote{"}

% Captions for figures
\captionsetup{justification=raggedright, format=plain, font=small,labelfont=bf}

\newcommandx{\unsure}[2][1=]{\todo[linecolor=orange,backgroundcolor=orange!25,bordercolor=orange,#1]{#2}}
\newcommandx{\Todo}[2][1=]{\todo[linecolor=yellow,backgroundcolor=yellow!25,bordercolor=yellow,#1]{#2}}
\newcommandx{\info}[2][1=]{\todo[linecolor=green,backgroundcolor=green!25,bordercolor=green,#1]{#2}}

% Centered, equally spaced columns
\newcolumntype{Y}{>{\centering\arraybackslash}X} % centered equidistant columns

\setenumerate{label=(\arabic*),itemsep=0mm} 

\renewcommand{\epsilon}{\varepsilon}

\DeclareMathOperator*{\argmax}{arg\,max}
\DeclareMathOperator*{\argmin}{arg\,min}
\DeclareMathOperator{\JSD}{JS}
\DeclareMathOperator{\KL}{KL}
\newcommand{\T}{\mathrm{T}}
\newcommand{\R}{\mathbb{R}}
\newcommand{\cX}{\mathcal{X}}
\newcommand{\cG}{\mathcal{G}}
\newcommand{\cD}{\mathcal{D}}
\newcommand{\cF}{\mathcal{F}}
\newcommand{\cO}{\mathcal{O}}
\newcommand{\cZ}{\mathcal{Z}}
\newcommand{\cB}{\mathcal{B}}
\newcommand{\N}{\mathbb{N}}
\newcommand{\E}{\mathbb{E}}
\newcommand{\Z}{\mathbb{Z}}
\newcommand{\fL}{\mathfrak{L}}
\newcommand{\RL}{\textsc{Ring Loading}\xspace}
\newcommand{\RRL}{\textsc{Relaxed Ring Loading}\xspace}
\newcommand{\dOne}{\mathbbm{1}}

% boxed problem environment
\newenvironment{problem}[1]
{
%	\noindent\begin{fbox}
%		\begin{parbox}{\textwidth}
			\begin{center}
				\textsc{#1}
			\end{center}
}
{

%		\end{parsbox}
%	\end{fbox}
}
			    % restliche Befehle und Pakete

%%%%%%%%%%%%%%%%%%%%%%%%%%%%%%%%%%%%%%%%%%%%%%%%%%%%%%%%%%%%%%%%%%%%%%%%%%%%%%%%%%%%%%%%%%%
%%%%%%%%%%%%%%%%%%%%%%%%%%%%%%%%%%%%%%%%%%%%%%%%%%%%%%%%%%%%%%%%%%%%%%%%%%%%%%%%%%%%%%%%%%%
% Start des Dokuments
\begin{document}		

\selectlanguage{english}
%\ownautorefnames		% �nderung einiger automatischen Texte von hyperref (wie in referenz.tex definiert)
\parindent0em 			% kein Einzug nach einer Leerzeile

%%%%%%%%%%%%%%%%%%%%%%%%%%%%%%%%%%%%%%%%%%%%%%%%%%%%%%%%%%%%%%%%%%%%%%%%%%%%%%%%%%%%%%%%%%%
% Titelseite
\thispagestyle{empty}		% leerer Seitenstil, also keine Seitennummer auf der Titelseite
\begin{titlepage}
\seminarkopf 			% Titelblatt (wie in kopf.tex definiert)
\begin{abstract}
	The ring loading problem arises during the planning of SONET rings.
	Given a cycle graph of $n > 1$ nodes and a traffic demand for each pair of nodes, the problem is to determine a routing in which each demand is routed either completely in the clockwise or completely in the counterclockwise direction around the ring, while minimizing the maximal load on any edge.
	
	Since this problem is NP-complete, Schrijver, Seymour and Winkler presented an approximation algorithm in their paper "The Ring Loading Problem".
	This algorithm computes approximations that exceed the optimal solution by at most $\frac{3}{2}D$, where $D$ is the maximal demand.
	In the process, an optimal solution to the relaxed ring loading problem is determined, which allows demands to be routed both ways at the same time.
	
	In this work, Schrijver et al.'s algorithm is presented and its is correctness proven.	
\end{abstract} 
\end{titlepage}

%%%%%%%%%%%%%%%%%%%%%%%%%%%%%%%%%%%%%%%%%%%%%%%%%%%%%%%%%%%%%%%%%%%%%%%%%%%%%%%%%%%%%%%%%%%
% Inhaltsverzeichnis
\thispagestyle{empty}	
\tableofcontents		% Inhaltsverzeichnis
%\listoffigures			% Abbildungsverzeichnis (eventuell einf�gen)
%\listoftables			% Tabellenverzeichnis (eventuell einf�gen)
\setcounter{page}{0}% Eigentlicher Inhalt beginnt auf Seite 1
\clearpage          % neue Seite f�r eigentlichen Inhalt
%%%%%%%%%%%%%%%%%%%%%%%%%%%%%%%%%%%%%%%%%%%%%%%%%%%%%%%%%%%%%%%%%%%%%%%%%%%%%%%%%%%%%%%%%%%
% Eigentlicher Inhalt der Seminararbeit; die einzelnen Teile werden hier (aus Gr�nden der �bersichtlichkeit) �ber \input{file} eingebunden

\section{Introduction}

The ring loading problem first arose in the planning of \textbf{S}ynchronous \textbf{O}ptical \textbf{Net}working (SONET) rings.
In SONET, several add-drop multiplexers (ADMs) and other components are connected by mostly optical fibers.
One prominent SONET topology is the bidirectional path-switched ring, which is safe to the failure of a single link or node due to its connectivity.
In a SONET ring, the routing of node-to-node network traffic is fixed: The traffic between any two nodes is either always routed in the clockwise or the counter-clockwise direction.
The ring loading problem arises in the planning of SONET rings.
ADMs have a limited capacity of traffic they can send or receive through any link.
Hence the maximal workload on any link must be determined from estimated node-to-node network traffic demands during planning.
This involves computing the optimal routing of these demands around the ring.
As it turns out, determining such a routing is quite difficult.

In their paper "The Ring Loading Problem" from 1999, the Schrijver, Seymour and Winkler~\cite{schrijver99} derived an approximation algorithm for the ring loading problem.
Parts of this paper are presented and discussed here.

We begin by formalizing the problem.
The SONET ring is modeled by an undirected cycle graph consisting of $n \in \N$ nodes.
We number the nodes consecutively in the counter-clockwise direction with the numbers $1, 2, \dots, n$.
For the ease of notation, it makes sense to interpret the node numbers as representatives of the equivalence classes of $\Z/n\Z$.
This means that we can write the edge set as $E = \{\{i, i+1\}\ |\ i \in [n]\}$.
Now, for every two nodes $i \neq j$ we are given a (traffic) demand $d_{i, j} \in \R_{\geq 0}$, which represents the traffic that has to be routed through from node $i$ to node $j$.
In practice, the demands from $i$ to $j$ and $j$ to $i$ can differ.
However, XYZ have shown that we can without loss of generality combine these two demands into one single demand from $i$ to $j$ and still obtain the same optimal routing.
Thus, we focus on the case where for every two nodes $i < j$, we have one demand $d_{i,j}$.

A demand $d_{i, j}$ can be routed two ways around the ring.
We say that a demand is routed \emph{forward} or \emph{through the front}, if it does not use the edge $\{n, 1\}$.
Otherwise, we say that the demand is routed \emph{backwards} or \emph{through the back}.

Next, we can formally define routings in a ring of size $n$.
\begin{definition}[Routing]
	\label{def:routing}
	A \emph{real routing} is a function 
	\begin{equation}
		\Phi: \{(i, j)\ | \ 1 \leq i < j \leq n\} \rightarrow [0, 1] \ .
	\end{equation}
	A \emph{binary routing}, or simply \emph{routing}, is a real routing $\Phi$ only takes values in $\{0, 1\}$. 
\end{definition}
We interpret the value $\Phi(i, j)$ as the fraction of the demand $d_{i, j}$ that is routed through the front.

\begin{definition}[Edge load, Ringload]
	\label{def:edge-load}
	Let $\Phi$ be a (real) routing.
	For every edge $\{k, k+1\}$, $k \in [n]$, we define the \emph{load} $L_k$ as the total traffic that is routed through that edge:
	\begin{equation}
		\label{eq:edge-load}
		L_k(\Phi) \coloneqq \sum_{\substack{i < j\\ k \in [i, j)}} d_{i, j} \Phi(i, j) + \sum_{\substack{i < j\\ k \notin [i, j)}} d_{i, j} (1 - \Phi(i, j)) \ .
	\end{equation}
	The \emph{ringload} $L(\Phi)$ is the maximal load under the routing $\Phi$.
\end{definition}
In \cref{def:edge-load}, we have $k \in [i, j)$ if and only if the edge $\{k, k+1\}$ lies on the front route of the demand $d_{i, j}$.


With these preparations, we can now formulate the ring loading problem in mathematical terms:
\begin{center}
	\begin{mdframed}
		\centering
		\textsc{Ring Loading} (\textsc{RL})\\[0.7em]
		\begin{tabular}{rl}
			{\bfseries Input}: & Ring size $n \in \N$ and demands $d_{i, j}$ for all $1 \leq i<j\leq n$.\\
			{\bfseries Output}: & Binary routing $\Phi$ that minimizes $L = \max_{i \in [n]} L_i(\Phi)$.
		\end{tabular}
	\end{mdframed}
\end{center}
If we omit any zero demands, we can encode an instance of \RL in (approximately) $\cO(k)$ space, where $k \in N$ is the number of non-zero demands.

\RL can be formulated 
As mentioned above, \RL is an NP-complete problem.
\begin{theorem}
	\RL is NP-complete.
\end{theorem}
\begin{proof}
	A routing provides a 
\end{proof}



Already shown: Tight bound $L^\mathrm{opt} \leq 2 L^\ast$ (take references from däubel intro!! TODO).

\citet{schrijver99} achieve an additive bound of $L^\mathrm{opt} + \frac{3}{2}D$.
This bound has since been improved to $1.3$ but the improvement based on the same algorithm.
It is therefore still sensible to review and understand the algorithm presented by \citet{schrijver99} in the late 1990s.

The authors of \cite{schrijver99} claim that their algorithm runs in $O(k n^2)$, where $k$ is the number of non-zero demands.

In this work, I show that the ideas presented by \citet{schrijver99} can be extended to obtain an algorithm that runs in $\cO(n^2)$.
This is almost optimal, as the encoding length of an instance is in $\Theta(k)$ and in the worst case $k = \binom{n}{2} \in \cO(n^2)$.
The modified algorithm strongly relies on the principles of dynamic programming and uses a different, more efficient way of routing demands.


This seems to be a novel result which is interesting for several reasons:
\begin{itemize}
	\item The best currently known runtime for computing an $L^\mathrm{opt} + \frac{3}{2}D$ approximation to \RL is $O(k n^2)$.
	The proposed algorithm improves this runtime by a factor of $k$.
	\item The best currently known algorithm for solving \RRL with integer demands is $O(k + t_k)$ for integer demands, where $t_k$ is the time for sorting $k$ integers.
	For real demands, the best known runtime is $O(\min\{kn, n^2\})$.
	The proposed algorithm also works for real demands and almost matches the runtime of the latter algorithm.
	\item The proposed algorithm computes the same solution to \RL and \RRL as Schrijver et al.'s algorithm, which is particularly aesthetic in the sense that at most $\lfloor\frac{n}{2}\rfloor$ demands are split while the others are routed all front or all back.
	This means that the runtimes of all algorithms that use Schrijver et al.'s algorithm as a subroutine are improved as well, in particular that of the $(1 + \epsilon)$ algorithm by \citet{khanna97}.
\end{itemize}


%\section{Ring Graphs and Terminology}

"Cycle Graph"

A \emph{ring of size $n$} is an undirected graph with the vertex set that corresponds to the integer equivalence classes modulo $n$ and edges $\{[i], [i+1]\}$ for $1 \leq i \leq n$.
For simplicity we write $i$ instead of $[i]$ and $\{i, i+1\}$ for the $i$-th edge. 

Ring as weighted chords + image.

Ring Load: maximum of link loads.

$[n]$ is the standard $n$-set $\{1, 2, \ldots, n\}$.

%
%\section{Ring Loading Problems and Complexity}

Definition of (Integer) Ring Loading also as decision problem

Definition of Relaxed Ring Loading also as decision problem

Complexity of Ring Loading and Relaxed Ring Loading

Interval-notation for ring interpreted as equivalence classes of finite ring.

\begin{notation}
	Let $1 \leq i < j \leq n$.
	Then we write $[i, j) \coloneqq \{i, i+1, \ldots, j-2, j-1\}$.
	We also define $[j, i) \coloneqq \{j, j+1, \ldots, n-1, n, 1, \ldots, i-2, i-1\}$.
	Open, half-open and closed intervals are to be interpreted in the same manner as real intervals.
\end{notation}

\begin{problem}{Ring Loading}
	\textbf{In:} Ring of size $n \in \N$, demands $d_{ij} \in \R^+$ for $1 \leq i < j \leq n$.\\
	\textbf{Goal:} Find a function $\Phi: \{(i, j)~|~1 \leq i < j \leq n\} \rightarrow \{0, 1\}$ such that
	$\max_{1 \leq i \leq n} L_i$ is minimal, where
	\begin{equation}
		L_i(\phi) \coloneqq \ldots
	\end{equation}
\end{problem}
Ring loading is a routing problem where traffic must be routed either through the front or the back.

$\phi(i, j) = 1$ means that the traffic from the demand $d_{ij}$ is routed through the path $\{i, i+1, \ldots, j-1, j\}$.
This route is called the \emph{front route}.
Otherwise, if $\phi(i, j) = 0$, the traffic is routed through $\{j, j+1, \ldots, n-1, n, 1, \ldots, i\}$, which is called the \emph{back route}.
The back route always contains the link $\{n, 1\}$.

\begin{theorem}
	\RL in its decision form is NP-complete.
\end{theorem}
\begin{proof}
	\Todo{Input can be encoded in O(n**2) space}
	A function $\Phi$ of the desired form can be encoded using $\cO(n^2)$ space, e.g. as a binary array, and serves as witness for \RL.
	This implies that \RL is in NP.
	In order to show NP-completeness, we provide a polynomial-reduction of the \textsc{Partition} problem \cite{karp72}.
	Given $m$ integers $\{z_1, \ldots, z_m\}$, this problem asks whether there exists a set $S \subseteq [m]$ such that 
	\begin{equation}
		\sum_{i \in S} z_i = \sum_{j \in [m] \setminus S} z_j \ .
	\end{equation}
	
\end{proof}

\begin{definition}
	Let $1 \leq g < h \leq n$.
	A \emph{cut} $\{g, h\}$ is a set of two links $\{\{g, g+1\}, \{h, h+1\}\}$.
	A demand $d_{ij}$ is said to \emph{cross} a cut $\{g, h\}$ if exactly one of $i$ and $j$ lies within $[g, h)$.
	The \emph{total demand across $\{g, h\}$} is defined as
	\begin{equation}
		D_{gh} \coloneqq \sum_{\{d_{ij}~|~d_{ij}\ \text{crosses}\ \{g, h\}\}} d_{ij} \ .
	\end{equation}
	This is the sum of the demands that must cross either $\{g, g+1\}$ or $\{h, h+1\}$.
\end{definition}

\begin{figure}
	\centering
	\begin{minipage}[t]{.5\textwidth}
		\begin{tikzpicture}[font=\scriptsize, node/.style={circle,thick,draw},
		l_2/.style={line width =0.25mm},
		scale=1, transform shape]
			% equidistant points and arc
			\foreach \x [count=\p] in {0,...,7} {
				\node[shape=circle,fill=black, scale=0.5] (\p) at (\x*45-135:2) {};
			};
			\foreach \x [count=\p] in {0,...,7} {
				\draw (225 + \x*45:1.7) node {\p};
%				\draw (-30-\x*60:2.4) node {$\bar{\p}$};
			}; 
			\draw[l_2] (4) arc (0:360:2);
			
			\draw[line width=0.4mm, red!90!black] (67.5:3.3) to [out=-112.5,in=67.5] (67.5:2) to [out=-112.5,in=112.5] (-67.5:2) to [out=-67.5, in=112.5](-67.5:3.3);
			\node (cut1) at (58:3) {$\{2, 5\}$};
			
			\draw[line width=0.4mm, red!90!black] (157.5:3) to [out=-22.5,in=157.5] (157.5:2) to [out=-22.5,in=22.5] (-157.5:2) to [out=-157.5, in=22.5](-157.5:3);
			\node (cut2) at (164:3.1) {$\{7, 8\}$};
			
			\node (a) at (-22.5:3) {$d_{2, 5}$};
			\draw[<->] (2)  to [out=-22.5,in=-112.5] (-22.5:2.5) to [out=67.5,in=-22.5](5);
%			\node (b) at (-67.5:2.8) {$d_{1, 4}$};
%			\draw[<->] (1)  to [out=-67.5,in=-157.5] (-67.5:2.5) to [out=22.5,in=-67.5] (4);
			
			\node (bottom) at (0, -2.8) {};
			%		\draw[dashed] (1) -- (3) -- (5) -- (1);
			% axes
			%		\draw [dotted, gray] (-2.6,0) -- (2.6,0);
			%		\draw [dotted, gray] (0,-2.15) -- (0,2.15);
		\end{tikzpicture}
	\end{minipage}
	\caption{Examples of cuts (red).
	We imagine a cut as a chord connecting the midpoints of its edges.
	The demand $d_{2, 5}$ crosses the cut $\{2, 5\}$ and is parallel to the cut $\{7, 8\}$.
	The cuts $\{2, 5\}$ and $\{7, 8\}$ are parallel.}
	\label{fig:cut-example}
\end{figure}

We can formulate a relaxed version of \RL, which allows demands to be routed both ways around the ring.
\begin{problem}{Relaxed Ring Loading}
	\textbf{In:} Ring of size $n \in \N$, demands $d_{ij} \in \R^+$ for $1 \leq i < j \leq n$.\\
	\textbf{Goal:} Find a function $\phi: \{(i, j)~|~1 \leq i < j \leq n\} \rightarrow [0, 1]$ such that $\max_{1 \leq i \leq n} L_i^\ast$ is minimal, where
	\begin{equation}
		L_i^\ast(\phi) \coloneqq \ldots
	\end{equation}
\end{problem}
In contrast to its binary version, \RRL can be solved in polynomial time, as it can be formulated as the following linear problem
\begin{alignat}{2}
	&\min &\quad& L\\
	&s.t. &\quad& L \geq L_i^\ast(\Phi)\quad \forall 1 \leq i \leq n \ .
\end{alignat}
Note that the $L_i$ are linear functions.


\section{An Algorithm for Relaxed Ring Loading}
\label{sec:relaxed-ring-loading}

In this section, Schrijver et al.'s \cite{schrijver99} algorithm for \RRL is presented.
Remarkably, this algorithm computes a real-valued routing $\Phi$ in which all but at most $\lfloor \frac{n}{2} \rfloor$ demands are already routed all front or all back.
This makes it easy to approximate an optimal binary routing, as will be shown later.

\subsection{Minimal Solutions}

We aim to determine a minimal routing:

\begin{definition}
	Let $\Phi^\ast$ be an optimal solution to an instance of \RRL.
	We say that $\Phi^\ast$ is \emph{minimal} if there exists no other optimal routing $\Phi'$ such that $L_i(\Phi') \leq L_i(\Phi^\ast)$ for all $1 \leq i \leq n$ and $L_j(\Phi') < L_j(\Phi^\ast)$ for at least one $j$.
\end{definition}

Minimal routings have the desired property of most demands being routed all front or all back.
\begin{theorem}
	\label{theo:number-of-splits}
	Let $\Phi^\ast$ be a minimal solution to an instance of \RRL of size $n$.
	Then $\Phi^\ast$ splits at most $\lfloor \frac{n}{2} \rfloor$ demands.
\end{theorem}

This result is quite remarkable, as we can have up to $\binom{n}{2} = \frac{n(n-1)}{2} \in \Theta(n^2)$ non-zero demands.
Before proving the theorem, it is sensible to introduce some additional notation and terminology.
Also, from now on we consider the \RRL instance $I$ of size $n \in \N$ as fixed.
\begin{definition}
	Let $i, j, u, v \in [n]$, $i < j, u < v$ and $d_{i,j}, d_{u, v}$ be two demands.
	We say that
	\begin{enumerate}
		\item $d_{i, j}$ and $d_{u, v}$ are \emph{crossing} if $i, j, u, v$ are pairwise distinct and $i < u < j < h$ or $u < i < h < j$;
		\item $d_{i, j}$ and $d_{u, v}$ are \emph{parallel} if they are not crossing.
		\item Let $d_{i, j}$ and $d_{u, v}$ be parallel with $j < v$.
		The edge $\{k, k+1\}$ is \emph{in between} $d_{i,j}$ and $d_{u, v}$ if it lies on the front route of exactly one of the demands (in case $i \geq u$) or if it lies on the back routes of both demands (in case $j \leq u$).
	\end{enumerate}
\end{definition}
These definitions, especially that of edges in between parallel demands, might seem complicated, but can be more easily understood visually.
See \cref{fig:parallel-demands} for some examples.

\begin{figure}
	\centering
	\begin{subfigure}[t]{.4\textwidth}
		\begin{tikzpicture}[font=\scriptsize, node/.style={circle,thick,draw},
		l_2/.style={line width =0.25mm},
		scale=1, transform shape]
		% equidistant points and arc
		\foreach \x [count=\p] in {0,...,7} {
			\node[shape=circle,fill=black, scale=0.5] (\p) at (\x*45-135:2) {};
		};
		\foreach \x [count=\p] in {0,...,7} {
			\draw (225 + \x*45:1.7) node {\p};
			%				\draw (-30-\x*60:2.4) node {$\bar{\p}$};
		}; 
		\draw[l_2] (4) arc (0:360:2);
		\node (a) at (-22.5:3) {$d_{2, 5}$};
		\draw[<->] (2)  to [out=-22.5,in=-112.5] (-22.5:2.5) to [out=67.5,in=-22.5](5);
		\node (b) at (-67.5:2.8) {$d_{1, 4}$};
		\draw[<->] (1)  to [out=-67.5,in=-157.5] (-67.5:2.5) to [out=22.5,in=-67.5] (4);
		
		\node (bottom) at (0, -2.8) {};
		%		\draw[dashed] (1) -- (3) -- (5) -- (1);
		% axes
		%		\draw [dotted, gray] (-2.6,0) -- (2.6,0);
		%		\draw [dotted, gray] (0,-2.15) -- (0,2.15);
		\end{tikzpicture}
		\subcaption{$d_{1, 4}$ and $d_{2, 5}$ are crossing.}
	\end{subfigure}
%	\hspace{1cm}
	\begin{subfigure}[t]{.4\textwidth}
		\captionsetup{width=1.1\linewidth}
		\begin{tikzpicture}[font=\scriptsize, node/.style={circle,thick,draw},
		l1_green/.style={thick, green!80!black},
		l1_red/.style={thick, blue!80!white},
		l_2/.style={},
		l_3/.style={line width =0.25mm},
		scale=1, transform shape]
		\draw[l_3] (2, 0) arc (0:360:2);
		
		\draw[l1_green] (5) arc (45:90:2);
		\draw[l1_green] (8) arc (180:270:2);
		\draw[l1_red] (4) arc (0:45:2);
		% equidistant points
		\foreach \x [count=\p] in {0,...,7} {
			\node[shape=circle,fill=black, scale=0.5] (\p) at (\x*45-135:2) {};
		};
		% labels
		\foreach \x [count=\p] in {0,...,7} {
			\draw (225 + \x*45:1.7) node {\p};
			%				\draw (-30-\x*60:2.4) node {$\bar{\p}$};
		};
		
		\node (a) at (10:2.9) {$d_{2, 5}$};
		\draw[l_2, <->] (2)  to [out=-22.5,in=-112.5] (-22.5:2.8) to [out=67.5,in=-22.5](5);
		
		\node (b) at (-15:2.46) {$d_{2, 4}$};
		\draw[l_2,<->] (2)  to [out=-15,in=-135] (-45:2.25) to [out=45,in=-75] (4);
		
		\node (c) at (135:2.7) {$d_{6, 8}$};
		\draw[l_2, <->] (6)  to [out=157.5,in=45] (135:2.4) to [out=-135,in=112.5] (8);
		
		\node (bottom) at (0, -2.8) {};
		%		\draw[dashed] (1) -- (3) -- (5) -- (1);
		% axes
		%		\draw [dotted, gray] (-2.6,0) -- (2.6,0);
		%		\draw [dotted, gray] (0,-2.15) -- (0,2.15);
		\end{tikzpicture}
		\subcaption{$d_{2,4}$, $d_{2, 5}$ and $d_{6, 8}$ are pairwise parallel.}
	\end{subfigure}
	\caption{Examples of crossing and parallel demands.
		In (b), the green edges lie in between $d_{2, 5}$ and $d_{6, 8}$.
		The blue edge is the only edge in between $d_{2, 4}$ and $d_{2, 5}$.}
	\label{fig:parallel-demands}
\end{figure}

%\begin{notation}
%	Let $i, j \in \N$ with $i < j$.
%	We write $[i, j)$ for the set $\{i, i+1, \ldots, j-1\}$.
%\end{notation}

For the proof of \cref{theo:number-of-splits}, we require the following lemma.
\begin{lemma}
	\label{lemma:parallel-demands}
	Let $\Phi^\ast$ be a minimal routing and $d_{i, j}$ and $d_{u, v}$ be parallel demands.
	Then no edge in between $d_{i, j}$ and $d_{u, v}$ carries traffic from both demands.
\end{lemma}
\begin{proof}
	Let $\Phi^\ast$ be a minimal solution and assume that the edge $\{k, k+1\}$ lies in between two parallel demands $d_{i, j}, d_{u, v}$ and carries traffic from both demands.
	Let $\alpha$ and $\beta$ be the amounts of traffic from $d_{i,j}$ and $d_{u, v}$, respectively, that are routed through $\{k, k+1\}$.
	Furthermore, assume w.l.o.g. that $\alpha \leq \beta$ and that $\{k, k+1\}$ lies on the front route of $d_{i, j}$ (the other cases can be reasoned analogously).
	
	We now construct a new routing $\Phi'$ from $\Phi^\ast$ by rerouting traffic $\alpha$ of both demands such that it no longer passes through $\{k, k+1\}$.
	This results in $d_{i, j}$ being routed all front or all back.
	The rerouting of $d_{i, j}$ causes the loads on all edges on its front route to decrease by $\alpha$ and those on the back route to increase by $\alpha$.
	Similarly, the rerouting of $d_{u, v}$ increases the loads on all edges on its back route by $\alpha$ and decreases those on its front route by $\alpha$.
	
	Let $\{l, l+1\}$ be any edge.
	Then we can write the load under the new routing as
	\begin{equation}
	L_l(\Phi') = L_l(\Phi^\ast) 
	+ \alpha (\dOne_{\{l \notin [i, j)\}} - \dOne_{\{l \in [i, j)\}} +  \dOne_{\{l \in [g, h)\}} - \dOne_{\{l \notin [g, h)\}} )
	\end{equation}
	Here, $\dOne_{\{\ldots\}}$ is the indicator function that is $1$ if the condition in the index is true and $0$ else.
	We assumed that $\{k, k+1\}$ lies on the front route of $d_{i, j}$, but also that it lies in between $d_{i, j}$ and $d_{g, h}$.
	This implies that $[g, h] \subset [i, j]$.
	Hence $l \notin [i, j)$ and $l \in [g, h)$ cannot both be true at the same time.
	This shows that $L_l(\Phi') \leq L_l(\Phi^\ast)$.
	Furthermore, $k \in [i, j)$ and $k \notin [g, h)$ by definition of $k$.
	Thus we have $L_k(\Phi') < L_k(\Phi^\ast)$.
	Altogether, the existence of the routing $\Phi'$ contradicts the minimality of $\Phi^\ast$.
\end{proof}

We can now conduct the proof of \cref{theo:number-of-splits}.
\begin{proof}[Proof of \cref{theo:number-of-splits}]
	Let $\Phi^\ast$ be a minimal routing and let $S \coloneqq \{(i, j)\ |\ 0 < \Phi^\ast(i, j) < 1\}$ be the set of indices of demands that are split by $\Phi^\ast$.
	Furthermore, let $(i, j), (u, v) \in S$.
	
	If the demands $d_{i,j}$ and $d_{u, v}$ were parallel, there would be a link $\{k, k+1\}$ that carries traffic from both demands, as they are both split.
	However, it follows from the minimality of $\Phi^\ast$ and \cref{lemma:parallel-demands} that this cannot be the case, which implies that $d_{i,j}$ and $d_{u, v}$ are crossing.
	This requires, by definition, that $i, j, u, v$ are mutually distinct.
	
	This implies that every element of $[n]$ can occur in at most one tuple in $S$.
	As the tuples in $S$ each consist of two elements, there can be at most $\lfloor\frac{n}{2}\rfloor$ such tuples.
\end{proof}

The existence of minimal solutions is not clear, a priori.
In the following, we will show that minimal solutions always exist by providing an algorithm.

\subsection{Relaxed Ring Loading with Capacities}

In order to construct a minimal solution, we first consider a generalization of \RRL by appending an edge capacity $C_k \in \R_{\geq 0}$ for each edge $\{k, k+1\}$.
The resulting problem, \textsc{RelaxedRingLoadingWithCapacities} (\RRLWC), is to determine a real-valued routing $\Phi^\ast$ such that $L_k(\Phi^\ast) \leq C_k$ for all $k \in [n]$.
We denote instances of this problem by $(I, C)$, where $I$ is an \RRL instance and $C = (C_k)_{k \in [n]}$ the capacities.
\RRLWC does not always have feasible solutions.
In the following, we show a necessary and sufficient condition for the existence of feasible solutions.

\begin{definition}
	A \emph{cut} $\{g, h\}$ is a set of two distinct edges $\{g, g+1\}$, $\{h, h+1\}$.
\end{definition}

We can imagine a cut $\{g, h\}$ as a chord connecting the midpoints of the edges $\{g, g+1\}$ and $\{h, h+1\}$.
A cut therefore "splits" the ring into two parts.

\begin{definition}
	Let $d_{i, j}$ be a demand and $\{g, h\}$ a cut with $g < h$.
	\begin{enumerate}
		\item We say that $d_{i, j}$ \emph{crosses} $\{g, h\}$ if the endpoints of $d_{i, j}$ are on both sides of the cut.
		Formally, this is the case if and only if $\abs{\{i, j\} \cap (g, h]} = 1$.
		\item $d_{i, j}$ is \emph{parallel} to $\{g, h\}$ if it does not cross $\{g, h\}$.
		\item Two cuts are \emph{crossing} or \emph{parallel} in the same sense two demands are.
		\item The \emph{demand across the cut} $\{g, h\}$ is the sum of demands with endpoints on both sides of the cut:
		\begin{equation}
			\label{eq:cut-demand-definition}
			D_{g,h} \coloneqq \sum_{d_{i,j} \text{ crosses } \{g, h\} } d_{i, j}\ .
		\end{equation}
	\end{enumerate}
\end{definition}

\begin{figure}
	\centering
	\begin{minipage}[t]{.5\textwidth}
		\begin{tikzpicture}[font=\scriptsize, node/.style={circle,thick,draw},
		l_2/.style={line width =0.25mm},
		scale=1, transform shape]
			% equidistant points and arc
			\foreach \x [count=\p] in {0,...,7} {
				\node[shape=circle,fill=black, scale=0.5] (\p) at (\x*45-135:2) {};
			};
			\foreach \x [count=\p] in {0,...,7} {
				\draw (225 + \x*45:1.7) node {\p};
%				\draw (-30-\x*60:2.4) node {$\bar{\p}$};
			}; 
			\draw[l_2] (4) arc (0:360:2);
			
			\draw[line width=0.4mm, red!90!black] (67.5:3.3) to [out=-112.5,in=67.5] (67.5:2) to [out=-112.5,in=112.5] (-67.5:2) to [out=-67.5, in=112.5](-67.5:3.3);
			\node (cut1) at (58:3) {$\{2, 5\}$};
			
			\draw[line width=0.4mm, red!90!black] (157.5:3) to [out=-22.5,in=157.5] (157.5:2) to [out=-22.5,in=22.5] (-157.5:2) to [out=-157.5, in=22.5](-157.5:3);
			\node (cut2) at (164:3.1) {$\{7, 8\}$};
			
			\node (a) at (-22.5:3) {$d_{2, 5}$};
			\draw[<->] (2)  to [out=-22.5,in=-112.5] (-22.5:2.5) to [out=67.5,in=-22.5](5);
%			\node (b) at (-67.5:2.8) {$d_{1, 4}$};
%			\draw[<->] (1)  to [out=-67.5,in=-157.5] (-67.5:2.5) to [out=22.5,in=-67.5] (4);
			
			\node (bottom) at (0, -2.8) {};
			%		\draw[dashed] (1) -- (3) -- (5) -- (1);
			% axes
			%		\draw [dotted, gray] (-2.6,0) -- (2.6,0);
			%		\draw [dotted, gray] (0,-2.15) -- (0,2.15);
		\end{tikzpicture}
	\end{minipage}
	\caption{Examples of cuts (red).
	We imagine a cut as a chord connecting the midpoints of its edges.
	The demand $d_{2, 5}$ crosses the cut $\{2, 5\}$ and is parallel to the cut $\{7, 8\}$.
	The cuts $\{2, 5\}$ and $\{7, 8\}$ are parallel.}
	\label{fig:cut-example}
\end{figure}

Examples for the definitions above can be found in \cref{fig:cut-example}.

\begin{definition}
	An cut $\{g, h\}$ is satisfies the \emph{cut condition} if
	\begin{equation}
		D_{g,h} \leq C_g + C_h \ .
	\end{equation}
	A \RRLWC instance $I$ \emph{satisfies the cut condition} if each cut does.
	$C_g + C_h - D_{g,h}$ is called the \emph{slack} of the cut $\{g, h\}$.
	A cut is \emph{tight} if its slack is $0$.
\end{definition}

The cut condition resembles the max-flow-min-cut theorem from the theory of network flows:
The maximum flow passing through any (source-target) cut must never be greater than the cut's capacity.
In the framework of \RRLWC, we get a similar result:

\begin{theorem}
	\label{theo:cut-condition}
	An instance $(I, C)$ of \RRLWC is feasible if and only if it satisfies the cut condition.
\end{theorem}
For the proof, we will need the following lemma.

\begin{lemma}
	\label{lemma:parallel-diagonal-cuts}
	Let $\{g, h\}$, $\{s, t\}$ be two parallel cuts and $d_{i, j}$ parallel to both cuts.
	Then
	\begin{equation}
		D_{s,g} + D_{t,h} \geq D_{g,h} + D_{s,t} + 2 d_{i,j} \ .
	\end{equation}
\end{lemma}
The proof can be conducted by showing that all demands that cross at least one of the cuts $\{g, h\}$, $\{s, t\}$ also cross as many of the cuts $\{s, g\}$, $\{t, h\}$.
This can be shown by a simple distinction of cases.

\begin{proof}[Proof of \cref{theo:cut-condition}]
	If in an instance $I$ a cut $\{g, h\}$ does not satisfy the cut condition, we have $D_{g, h} > C_g + C_h$.
	All demands contributing to $D_{g, h}$ must pass through any (or both) of the edges $\{g, g+1\}$ and $\{h, h+1\}$, since their endpoints are on both sides of $\{g, h\}$.
	However, this is impossible without violating at least one of the capacity constraints.
	
	Now, assume that there are instances that satisfy the cut condition, but are not feasible.
	Let $I$ be the smallest such instance in terms of the ring size $n$ and among those, the instance with the smallest number of non-zero demands.
	Note that $I$ then has at least one non-zero demand, as it would otherwise satisfy the cut condition.
	
	We now construct another non-solvable instance $I'$ with one less non-zero demand and show that this instance still satisfies the cut condition.
	This will contradict the minimality of $I$, implying that no such instance exists.
	
 	We begin by picking any non-zero demand $d_{i,j}$ with $j - i \geq 2$.
 	(If no such demand should exist a priori, we can always obtain one by renumbering the nodes through "rotating".)
 	Let $\{g, h\}$ be a cut that minimizes
 	\begin{equation}
 		\overline{\mu} \coloneqq \min_{\substack{ \{g', h'\} \text{ cut},\\ g, h \in [i, j)}}
 		C_{g'} + C_{h'} - D_{g', h'} \ .
 	\end{equation}
 	$\overline{\mu}$ is the minimal slack of any cut that lies on the front route of $d_{i, j}$.
	
	We construct a new instance $I'$ by routing $\frac{\mu}{2} \coloneqq \min(\frac{\overline{\mu}}{2}, d_{i, j})$ traffic of $d_{i,j}$ along the front route and the remaining $d_{i,j} - \frac{\mu}{2}$ along the back route (if $d_{i,j} < \frac{\overline{\mu}}{2}$, we send all traffic through the front).
	Especially, $\mu \leq \overline{\mu}$.
	We set the capacities of the new instance accordingly, that is
	\begin{equation}
		C_k' \coloneqq 
		\begin{cases}
			C_k - \frac{\mu}{2}, & \text{if } k \in [i, j)\\
			C_k - (d_{i,j} - \frac{\mu}{2}) & \text{if } k \notin [i, j)
		\end{cases} \ , \quad \forall k \in [n] \ .
	\end{equation}
	The demands remain the same with the exception of $d_{i,j}$, which is now $0$.
	The new instance is still not feasible; otherwise we could use any feasible routing for $I'$ to construct a solution to $I$.
	
	By construction, all cuts $\{s, t\}$ in $I'$ that lie on the front route of $d_{i,j}$ still satisfy the cut condition:
	\begin{equation}
		D_{s,t}' = D_{s,t} \leq C_{s} + C_{t} - \mu = (C_{s} - \frac{\mu}{2}) + (C_{t} - \frac{\mu}{2}) \stackrel{\mathrm{Def.}}{=} C_{s}' + C_{t}' \ .
	\end{equation}
	The first equality follows from the fact that $d_{i,j}$ does not contribute to $D_{s,t}$ and $D_{s,t}'$, the inequality from the choice of $\mu$.
	
	Similarly, all cuts $\{s, t\}$ crossed by $d_{i,j}$ satisfy the cut condition:
	Assume w.l.o.g. $s \notin [i, j)$, $t \in [i, j)$, then the following holds true:
	\begin{align}
		D_{s,t}' &= D_{s,t} - d_{i,j} \leq C_s + C_t - d_{i,j} \\
		&= \left(C_s - \left(d_{i,j} - \frac{\mu}{2}\right)\right) + \left(C_h - \frac{\mu}{2}\right) \stackrel{\mathrm{Def.}}{=} C_s' + C_t' \ .
	\end{align}
	Here, the first equality results from $d_{i,j}$ no longer contributing to the demand across the cuts it crosses.
	The inequality is simply the cut condition in $I$.
	
	Now, pick any cut $\{s, t\}$ which lies on the back route of $d_{i,j}$ and assume it violates the cut condition.
	This means that
	\begin{equation}
		\label{eq:dst-capacity-inequality}
		D_{s,t} = D_{s,t}' > C_s' + C_h' \stackrel{\mathrm{Def.}}{=} \left(C_s - \left(d_{ij} - \frac{\mu}{2}\right)\right) + \left(C_h - \left(d_{ij} - \frac{\mu}{2}\right)\right) \ .
	\end{equation}
%	or equivalently $D_{s,t} +  2(d_{i,j} - \frac{M}{2}) > C_s + C_h$.
	We now consider the cuts $\{s, g\}$ and $\{t, h\}$.
	We get the following inequality for the sum of demands across the cuts $\{s, g\}$ and $\{t, h\}$ in the original instance:
	\begin{align}
		D_{s,g} + D_{t,h} &\geq D_{g,h} + D_{s,t} + 2 d_{i,j} \label{eq:diagonal-cut-inequality}\\
		&> (C_g + C_h - \mu) +  \left(C_s - \left(d_{i,j} - \frac{\mu}{2}\right)\right) + \left(C_h - \left(d_{ij} - \frac{\mu}{2}\right)\right) + 2 d_{i,j} \label{eq:strict-capacity-inequality}\\
		&= (C_s + C_g) + (C_t + C_h) \ .
	\end{align}
	Here, \ref{eq:diagonal-cut-inequality} follows from \cref{lemma:parallel-diagonal-cuts}.
	The strict inequality results from the choice of $\{g, h\}$ and $\mu$ and \cref{eq:dst-capacity-inequality}.
	Altogether, this means that $D_{s,g} > C_s + C_g$ or $D_{t,h} > C_t + C_h$, showing that at least one of the cuts $\{s, g\}$, $\{t, h\}$ must have violated the cut condition in $I$; a contradiction to the choice of $I$.
	
	Thus all cuts in $I'$ satisfy the cut condition.
	This contradicts the minimality of $I$ which shows that no such instance can exist.
\end{proof}

With the help of \cref{theo:cut-condition} it is easy to determine the optimal ringload for an instance of \RRL.
\begin{corollary}
	\label{cor:ringload}
	Let $M \coloneqq \max_{1 \leq g < h \leq n} D_{g, h}$.
	Then the optimal ringload $L^\ast$ of an instance $I$ of \RRL is $L^\ast = \frac{M}{2}$.
\end{corollary}
\begin{proof}
	It is easy to see that the instance $I$ with capacities $C_k = \frac{M}{2}$, $k \in [n]$ satisfies the cut condition.
	Hence $L^\ast \leq \frac{M}{2}$.
	
	Let $\{g, h\}$ be a cut for which $D_{g, h}$ is maximal.
	Then in any feasible routing, at least $\frac{M}{2}$ traffic must be routed through either of the links $\{g, g+1\}$, $\{h, h+1\}$, implying that $L^\ast \geq \frac{M}{2}$.
%	Then $M = D_{g, h}$ traffic must pass through the edges $\{g, g+1\}$ and $\{h, h+1\}$, which means that the for at least one of the edges, the edge load must be $\geq \frac{M}{2}$. 
\end{proof}
\subsection{Determining a Minimal Solution}

We now have almost all the tools we need to formulate an algorithm.
Starting with a \RRL instance $I$, the basic idea is the following:
First, choose capacities such that (1) the cut condition is satisfied and (2) every solution complying with these capacities is minimal.
Then, route the demands all front or all back until only mutually crossing demands remain (we will see how this can be done in the following).
Finally, split the remaining demands using the strategy used in the proof of \cref{theo:cut-condition}.

Let $M \coloneqq \max_{1 \leq g < h \leq n} D_{g, h}$.
We construct the capacities $C = (C_k)_{k \in [n]}$ in an iterative fashion, starting at $k = 1$:
\begin{equation}
	\label{eq:minimal-capacities}
	C_k \coloneqq \max \left(\max_{1 \leq g < k}(D_{g, k} - C_g), \max_{k < h \leq n}(D_{k, h} - \frac{M}{2})\right) \ .
\end{equation}

We note that:
\begin{proposition}
	The instance $(I, C)$ is feasible.
\end{proposition}
This follows immediately from the first argument in the maximum in \cref{eq:minimal-capacities}.

\begin{lemma}
	\label{lemma:capacities-bounded}
	It holds that $C_k \leq \frac{M}{2}$ for all $k \in [n]$.
\end{lemma}
\begin{proof}
	Let $k \in [n]$.
	We distinguish the two cases in \cref{eq:minimal-capacities}:
	\begin{enumerate}[align=left]
		\item[Case 1: $C_k = \max_{1 \leq h < k}(D_{g, k} - C_g)$]{\mbox{}\\
			Then there exists some index $g$, $1 \leq g < k$, for which we have
			\begin{align}
				C_k &= D_{g, k} - C_g 
				\stackrel{\mathrm{Def. } C_g}{\leq} D_{g, k} - \left(\max_{g < h \leq n} D_{g,h} - \frac{M}{2}\right)\\
				&= (D_{g, k} - \max_{g < h \leq n} D_{g, h}) + \frac{M}{2} \ .
			\end{align}
			Because of $g < k \leq n$ we have $D_{g, k} - \max_{g < h \leq n} D_{g, h} \leq 0$, proving that $C_k \leq \frac{M}{2}$.
		}
		\item[Case 2: $C_k = \max_{k < h \leq n}(D_{k, h} - \frac{M}{2})$]{\mbox{}\\
			Because of $D_{k, h} \leq M$ for all $h > k$, it immediately follows that $C_k \leq M - \frac{M}{2} = \frac{M}{2}$.
		}
	\end{enumerate}
\end{proof}


\begin{theorem}
	\label{theo:routing-with-capacities-is-minimal}
	Any routing complying with the capacities defined in \cref{eq:minimal-capacities} is minimal.
\end{theorem}
\begin{proof}	
	Let $\Phi$ a feasible real-valued routing with link loads $L_k$.
	We have seen in \cref{cor:ringload}, the optimal ringload is $L^\ast = \frac{M}{2}$.
	Since we also know that $C_k \leq \frac{M}{2}$ from \cref{lemma:capacities-bounded}, $\Phi$ is an optimal solution.
	
	Assume that $\Phi$ is not minimal and let $\Phi'$ another routing with link loads $L_k'$ with $L_k' \leq L_k$ for all $k \in [n]$ and $L_l' < L_l$ for at least one $l \in [n]$.
	We now define new capacities $C_k' \coloneqq L_k'$ for $k \in [n]$.
	The routing $\Phi'$ obviously complies with these capacities, hence they are feasible.
	
	Now, let $h \in [n]$ be the smallest index for which $C_h' < C_h$.
	We show that there is a cut $\{g, h\}$ which violates the cut condition for the new capacities by distinction of the two terms responsible for the choice of $C_h$ in \cref{eq:minimal-capacities}.
	\begin{enumerate}[align=left]
		\item[Case 1: $C_h = \max_{1 \leq g < h}(D_{g,h} - C_g)$]{\mbox{}\\
			Especially, we then have $C_h'= L_h' < L_h \leq C_h = \max_{1 \leq g < h}(D_{g, h} - C_g)$ and by the choice of $h$ as the smallest index, it follows that $C_k' = C_k$ for all $k < h$.
			This means that for $g \in \argmax_{1 \leq g < h}(D_{g, h} - C_g)$ we get $C_h' < C_h = D_{g, h} - C_g = D_{g, h} - C_g'$, implying that $D_{g, h} > C_g' + C_h'$.
			Thus, the cut $\{g, h\}$ violates the cut condition for the new capacities.
		}
		\item[Case 2: $C_h = \max_{h < g \leq n}(D_{h, g} - \frac{M}{2})$]{\mbox{}\\
			Similarly to case 1 we have $C_h'= L_h' < L_h \leq C_h = \max_{h < g \leq n}(D_{h,g} - \frac{M}{2})$.
			This means that for $g \in \argmax_{h < g \leq n}(D_{h,g} - \frac{M}{2})$ we have $C_h' < D_{h,g} - \frac{M}{2}$.
			Using \cref{lemma:capacities-bounded} we then get 
			\begin{equation}
				C_h' + C_g' \leq C_h' + C_g \leq C_h' + \frac{M}{2} < D_{h, g} \ ,
			\end{equation}
			again showing that the cut $\{g, h\}$ violates the cut condition.
		}
	\end{enumerate}
	Altogether, this shows that the instance $(I, C')$ does not satisfy the cut condition, a contradiction.
	Thus there exists no such instance $\Phi'$, proving that $\Phi$ is minimal.
\end{proof}

All that is left now is to see how we can route demands given "minimal" capacities as in \cref{eq:minimal-capacities}.
For that, we make the following three observations.
\begin{lemma}
	\label{lemma:parallel-demands-cross-cut}
	Let $d_{i ,j}$ and $d_{u, v}$, $d_{i, j} \neq d_{u, v}$ be parallel demands, $\{g, g+1\}$ an edge in between and $\{g, h\}$ a cut.
	Then at most one of $d_{i ,j}$ and $d_{u, v}$ crosses $\{g, h\}$.
\end{lemma}
The proof of \cref{lemma:parallel-demands-cross-cut} can be conducted by a distinction of the position of the edge $\{h, h+1\}$.
\begin{lemma}
	\label{lemma:parallel-to-tight-cut}
	Let $d_{i, j}$ be parallel to a tight cut $\{g, h\}$.
	Then in any feasible routing, $d_{i, j}$ must be routed parallel to $\{g, h\}$.
\end{lemma}
\begin{proof}
	The cut $\{g, h\}$ being tight means that $D_{g, h} = C_g + C_h$, that is, the total traffic of all demands that cross $\{g, h\}$ already saturates the edges $\{g,g+1\}$ and $\{h, h+1\}$.
	$d_{i, j}$ does not cross ${g, h}$ and can thus not be routed through these edges.
	This means that $d_{i ,j}$ must be routed parallel to $\{g, h\}$.
\end{proof}
\begin{lemma}
	\label{lemma:edge-in-tight-cut}
	For each edge $\{g, g+1\}$ there exists an edge $\{h, h+1\}$ such that the cut $\{g, h\}$ is tight.
\end{lemma}
\begin{proof}
	Assume the edge $\{g, g+1\}$ is not contained in a tight cut.
	Let $\nu \coloneqq \min_{h \neq g} C_g + C_h - D_{g, h}$.
	Define $C_g' = C_g - \nu$ and $C_k' = C_k$ for all $k \in [n], k \neq g$.
	Then every cut still satisfies the cut condition for the capacities $C_k'$ and we have $C_g' < C_g$.
	But in the proof of \cref{theo:routing-with-capacities-is-minimal} we have shown that no such set of capacities can exist.
\end{proof}

Finally, we can formulate Schrijver et al.s' \cite{schrijver99} algorithm (\cref{algo:rrl}).
\begin{figure}[ht]
	\vspace*{-1em}
	\centering
%	\DontPrintSemicolon
	\begin{algorithm}[H]
		\begin{mdframed}[backgroundcolor=green!10,linecolor=white,innerleftmargin=25pt,leftmargin=-25pt,rightmargin=15pt]
			\KwIn{Ring size $n \in \N$, demands $d_{i,j} \in \R_{\geq 0}$ for all $i, j \in [n], i < j$.}
			\nl Compute demands across cuts $D_{i, j}$ and $M = \max_{1 \leq i < j \leq n} D_{i,j}$\;
			\nl Compute the capacities $C_k$ in \cref{eq:minimal-capacities}\;
			\nl\While{there exist parallel demands $d_{i,j}, d_{u, v}$}{
				Choose edge $\{g, g+1\}$ that lies in between $d_{i,j}$ and $d_{u, v}$\;
				Find a tight cut $\{g, h\}$\;
				Route $d_{i, j}$ or $d_{u, v}$ to miss the cut $\{g, h\}$\;
				Decrease capacities accordingly and adjust demands across cuts\;
			}
			\nl\ForEach{unrouted demand $d_{i, j}$}{
				Compute minimal slack $\mu$ on the front route of $d_{i, j}$\;
				Route $\min(d_{i, j}, \frac{\mu}{2})$ forward and the remainder backwards\;
				Decrease capacities accordingly and adjust demands across cuts\;
			}
%			\nl\Return{Routing $\Phi$}
		\end{mdframed}
		\caption{Schrijver et al.'s \cite{schrijver99} algorithm for computing a minimal solution to an instance of \RRL.}
		\label{algo:rrl}
	\end{algorithm}
\end{figure}

The basic idea of the algorithm is the following:
After computing the demands across the cuts and the capacities, the demands are routed all front or all back until only mutually crossing demands remain.
In each iteration of Step~3, an edge $\{g, g+1\}$ in between two parallel demands $d_{i, j}$, $d_{u, v}$ is selected.
We find a tight cut $\{g, h\}$, whose existence is guaranteed by \cref{lemma:edge-in-tight-cut}.
\cref{lemma:parallel-demands-cross-cut} then implies that at most one of the demands crosses $\{g, h\}$ and  by \cref{lemma:parallel-to-tight-cut} the routing of the non-crossing demand is uniquely determined.
After decreasing the edge capacities accordingly and adjusting the demands across the cuts, we are left with an instance of \RRLWC with one less non-zero demand.
It remains to see that in \cref{lemma:edge-in-tight-cut} still holds true for the new, smaller instance:

\begin{lemma}
	\label{lemma:tight-cuts-remain-tight}
	During the execution of Step 3 of \cref{algo:rrl}, the cut slacks are non-increasing.
\end{lemma}
\begin{proof}
	Let $C_k$, $k \in [n]$ be the capacities before an it an iteration of the while-loop (Step 3) of \cref{algo:rrl} and let $d_{i, j}$ be the demand that is routed during this iteration.
	Let $\{g, h\}$ be any cut and let $C_k'$ be the new capacities and $D_{g, h}'$ the new demand across $\{g, h\}$.
	We distinguish two cases:
	\begin{enumerate}[align=left]
		\item[Case 1: $d_{i, j}$ crosses $\{g, h\}$]{\mbox{}\\
			Then exactly one of the capacities $C_g$, $C_h$ was decreased by $d_{i, j}$.
			The slack of $\{g, h\}$ does not change:
			\begin{equation}
				C_g' + C_h' - D_{g, h}' = (C_g + C_h - d_{i, j}) - (D_{g, h} - d_{i, j}) = C_g + C_h - D_{g, h} \ .
			\end{equation}
		}
		\item[Case 2: $d_{i, j}$ is parallel to $\{g, h\}$]{\mbox{}\\
			Then the demand across $\{g, h\}$ does not change, but the capacities $C_g$, $C_h$ might have been decreased.
			Thus, the slack of $\{g, h\}$ does not increase. \qedhere
		}
	\end{enumerate}
\end{proof}

This shows that cuts that are tight during any iteration of Step 3 remain tight, which is especially the case for all cuts that are tight in the beginning, implying \cref{lemma:edge-in-tight-cut} does indeed hold true in each iteration.

In Step 4 of \cref{algo:rrl}, the remaining cuts are split using the technique from the proof of \cref{theo:cut-condition}.

This shows the correctness of Schrijver et al.'s \cite{schrijver99} algorithm for \RRL.
The authors claim that the algorithm's runtime is in $\cO(k n^2)$, where $k$ is the number of non-zero demands.
This is fairly easy to see:
Computing the demand across any cut can be done using \cref{eq:cut-demand-definition} in $\cO(k)$ time.
There are $\binom{n}{2}$ cuts, which puts the runtime of Step 1 in $\cO(k n^2)$.
Step 2 can be carried out in $\cO(n^2)$ using the definition of the capacities in \cref{eq:minimal-capacities}.
One can verify that each iteration of the Step 3 and 4 takes $\cO(n^2)$ time, meaning that both loops are also in $\cO(k n^2)$.

%With some modifications, however, this runtime can be improved to $\cO(n^2)$, as will later be discussed in \cref{sec:runtime}.

%	Once a cut becomes tight in \cref{algo:rrl}, it remains tight.
%\end{lemma}
%\begin{proof}
%	There are two points in \cref{algo:rrl} where the slack of a cut can change:
%	When the capacities are decreased and when the $D_{ij}$ change.
%	First, note that decreasing the capacities can only decrease slack.
%	This cannot un-tighten any cut.
%	Second, when any $D_{ij}$ changes it must be due to a crossing demand $d_{gh}$ being routed either way.
%	This means that the new demand across the cut is $D_{ij}' = D_{ij} - d_{gh}$.
%	However, in this case the capacity of either $\{i, i+1\}$ or $\{j, j+1\}$ is also decreased by $d_{gh}$, meaning that
%	$C_i' + C_j' = C_i + C_j - d_{gh}$.
%	In total we get $C_i' + C_j' - D_{ij}' = C_i + C_j - D_{ij}$, showing that the slack does not change.
%\end{proof}


\section{An Algorithm for Ring Loading}
\label{sec:ring-loading}

In this section, we see how we can transform the real-valued routing $\Phi$ determined by \cref{algo:rrl} into a binary routing that approximates the optimal routing.
Let $L_i = L_i(\Phi)$ be the edges loads induced by $\Phi$ and $S$ be the set of demands that are split.
In order to obtain a binary routing, we simply have to reroute the demands in $S$ all front or all back -- ideally without increasing the maximal edge load by too much. 
As we have seen, all demands in $S$ are mutually crossing and $\abs{S} \leq \lfloor \frac{n}{2} \rfloor$.
That means, each node is an endpoint of at most one demand.
For simplicity, we remove all nodes $i \in [n]$ that are not endpoints of any demand.
When rerouting any split demand, the loads on the two edges incident to $i$ change by the same amount.
Hence, we replace the edges $\{i-1, i\}$ and $\{i, i+1\}$ with a single new edge $\{i', i'+1\}$ and set 
\begin{equation}
	L_{i'} \coloneqq \max(L_{i-1}, L_i) \ .
\end{equation}
For rerouting the split demands, it suffices to consider this contracted instance $I'$ of size $2m$, where $m \coloneqq \abs{S}$.
We can rewrite $S$ in a simpler way as $S' = \{d_{i, i+m}\ |\ i \in [m]\}$.

For all $d_{i, i+m} \in S'$, let $x_i$, $y_i$ be the amount of traffic of $d_{i, i+m}$ that $\Phi$ routes through the front and the back, respectively.
Rerouting $d_{i, i+m}$, e.g. by routing it all front increases the the edge loads on the front route of $d_{i, i+m}$ by $y_i$, while those on the back route are decreased by $y_i$.
Similarly, routing $d_{i, i+m}$ all back increases the loads on the back route and decreases those in the front by $x_i$, respectively.

Now, assume that all demands in $S'$ have been routed either all front or all back.
For each $d_{i, i+m} \in S'$ we define
\begin{equation}
	z_i = \begin{cases}
		y_i, &\text{if } d_{i, i+m} \text{ is routed all front} \\
		-x_i, &\text{if } d_{i, i+m} \text{ is routed all back}
	\end{cases} \ ,
\end{equation}
that is, $z_i$ is the change of traffic on front route of $d_{i, j}$.
This allows us to formulate the total change of traffic on any edge $\{k, k+1\}$, $k \in [2m]$ and determine the new link load $L_k'$:
\begin{equation}
	\label{eq:load-change}
	L_k' = L_k + \sum_{\substack{i \in [m],\\ k \in [i, i+m)}} z_i - \sum_{\substack{i \in [m],\\ k \notin[i, i+m)}} z_i \ .
\end{equation}
Here, we again have $k \in [i, i+m)$ if and only if the edge $\{k,k+1\}$ lies on the front route of $d_{i, i+m}$.
We also note that for all $k \in [m]$
\begin{equation}
	\label{eq:load-increase-equality}
	L_k' - L_k 
	= \sum_{\substack{i \in [m],\\ k \in [i, i+m)}} z_i - \sum_{\substack{i \in [m],\\ k \notin[i, i+m)}} z_i 
	= L_{k + m} - L_{k + m}' \ .
\end{equation}

The following lemma shows that there is a way to reroute the demands in $S'$ such that the maximal edge load remains bounded.
\begin{lemma}
	\label{lemma:reroute-demands}
	Let $x_i, y_i \in \R_{\geq 0}, i \in [m]$ be such that $x_i + y_i \leq D$ for $D \in \R_{\geq 0}$.
	Then there exist $z_i \in \R$, $i \in [m]$ such that:
	\begin{align}
		(z_i = y_i \quad \text{ or } \quad z_i = -x_i) 
		\quad \text{ and } \quad \sum_{i=1}^k z_i \in \left[-\frac{D}{2}, \frac{D}{2}\right] \quad \forall k \in [m] \ .
	\end{align}
\end{lemma}
\begin{proof}
	We iteratively define the variables $z_i$.
	Set $z_1 \coloneqq y_1$.
	Then, for $k \in [n]$, let $z_i, 1 \leq i < k$ be such that $\sum_{i=1}^{k-1} z_i \in \left[-\frac{D}{2}, \frac{D}{2}\right]$.
	Since $x_k + y_k \leq D$, at least one of the inequalities $\sum_{i=1}^{k-1} z_i + y_k \leq \frac{D}{2}$, $\sum_{i=1}^{k-1} z_i - x_i \geq -\frac{D}{2}$ holds true.
	We choose $z_k = y_k$ or $z_k = -x_k$ accordingly.	
\end{proof}

We summarize the approximation quality of the rerouting in the following theorem.
\begin{theorem}
	\label{theo:ring-loading-algorithm}
	Let $\Phi$ be the routing obtained from \cref{algo:rrl} and $\Lopt$ be the ringload of an optimal binary routing.
	Then the demands that are split by $\Phi$ can be rerouted such that the resulting binary routing $\Phi'$ has a ringload of
	\begin{equation}
		L(\Phi') \leq \Lopt + \frac{3}{2}D \ ,
	\end{equation}
	where $D = \max_{1 \leq i < j \leq n} d_{i, j}$ is the largest demand.
\end{theorem}
\begin{proof}
	For all $d_{i, i+m} \in S'$, let $x_i$, $y_i$ be the amount of traffic of $d_{i, i+m}$ that $\Phi$ routes through the front and the back, respectively.
	Choose $z_i, i \in [m]$ like in \cref{lemma:reroute-demands} with $D = \max_{1 \leq i < j \leq n} d_{i, j}$.
	We construct the binary routing $\Phi'$ by setting $\Phi'(i, j) \coloneqq \Phi(i, j)$ for all $d_{i,j} \notin S$.
	For all demands $d_{i, j}$ that are split by $\Phi$, we set
	\begin{equation}
		\Phi'(i, j) = \begin{cases}
			1, & \text{if } z_i = y_i \\
			0, & \text{if } z_i = -x_i
		\end{cases}
	\end{equation}
	where $z_i$ corresponds to the demand $d_{i, j}$ in $S$.
	We get:
	\begin{equation}
		L(\Phi') - L(\Phi) 
		= \max_{k \in [n]} (L_k(\Phi') - L(\Phi)) 
		\leq \max_{k \in [n]} (L_k(\Phi') - L_k(\Phi)) \ .
	\end{equation}
	We observed that the link loads in the contracted instance change by the same amount as those in the original instance.
	This yields
	\begin{align}
		\max_{k \in [n]} (L_k(\Phi') - L_k(\Phi)) 
		&\stackrel{\mathclap{(\ref{eq:load-change})}}{=} \max_{k \in [2m]} \left(\sum_{\substack{i \in [m], k \in [i, i+m)}} z_i - \sum_{\substack{i \in [m], k \notin[i, i+m)}} z_i \right)\\
		&\stackrel{\mathclap{(\ref{eq:load-increase-equality})}}{=} \max_{k \in [m]} \abs{\sum_{\substack{i \in [m], k \in [i, i+m)}} z_i - \sum_{\substack{i \in [m], k \notin[i, i+m)}} z_i} \\
		& = \max_{k \in [m]} \abs{\sum_{i = 1}^k z_i - \sum_{i = k+1}^m z_i }\\
		& = \max_{k \in [m]} \abs{2\sum_{i = 1}^k z_i - \sum_{i = 1}^m z_i } \ . \label{eq:partial-reroutes}
	\end{align}
	Because both $\sum_{i = 1}^k z_i$ and $\sum_{i = 1}^m z_i$ lie in the interval $[-\frac{D}{2}, \frac{D}{2}]$, 
	we have 
	\begin{equation}
		2\sum_{i = 1}^k z_i - \sum_{i = 1}^m z_i \in [-\frac{3}{2}D, \frac{3}{2}D] \ .
	\end{equation}
	Thus we get $L(\Phi') - L(\Phi) \leq \frac{3}{2}D$, or equivalently $L(\Phi') \leq L(\Phi) + \frac{3}{2}D$.
	Observing that $L(\Phi) \leq \Lopt$ concludes the proof.
\end{proof}

This result completes the derivation of the approximation algorithm for \RL presented by \citet{schrijver99}.
The proof of \cref{lemma:reroute-demands} shows that the rerouting of split demands can be carried out in $\cO(n)$ time.
In total this puts the runtime of the algorithm in $O(k n^2 + n) = O(k n^2)$. 

%\section{Runtime Improvements}
\label{sec:runtime}

In this section I show that with some modifications, Schrijver et al.'s \cite{schrijver99} algorithm for \RRL (and therefore also \RL) can be improved to $\cO(n^2)$.

We do this by examining the steps of \cref{algo:rrl}.
In Step 1, the demands across all cuts are computed.
Doing this in the naïve way -- by using the definition in \cref{eq:cut-demand-definition} takes $O(n^2)$ time per cut, which means $O(n^4)$ time in total.
A more efficient way can be derived from the following lemma.
\begin{lemma}
	The following recursion holds true for all $g, h \in [n]$, $g < h$:
	\begin{equation}
		\label{eq:recursive-dac}
		D_{g, h} = \begin{cases}
			\sum_{k < g+1} d_{k,g+1} + \sum_{k > g+1} d_{g+1, k}, & \text{if } h = g+1 \\
			D_{g, g+1} + D_{g+1, h} - 2 A_{g+1, h}, & \text{if } h > g+1
		\end{cases} \ ,
	\end{equation}
	where the auxiliary variables $A_{g, h}$ are defined as:
	\begin{equation}
		A_{g, h} \coloneqq \begin{cases}
			d_{g, g+1}, & \text{if } h = g+1\\
			d_{g, h} + A_{g, h-1}, &\text{if } h > g+1\\
		\end{cases} \ .
	\end{equation}
\end{lemma}
%We note that \cref{eq:recursive-dac} is in fact a recursion, since in the case of $h > g+1$, only demands across "smaller" cuts are used.
\begin{proof}
	Let $\{g, h\}$, $g < h$ be a cut.
	If $h = g+1$, the demand across $\{g, h\}$ is exactly the sum of all demands with one of their endpoints equal to $g+1$.
	
	Now, let $h > g+1$ and let $B$ be the set of all demands with exactly one endpoint in $(g, h]$, meaning that $D_{g, h}$ is the sum over $B$.	
	In the following, let w.l.o.g $i \in (g, h]$.
	We rewrite $B$ as: 
	\begin{align}
		B &= \{d_{i, j}\ |\ \abs{\{i, j\} \cap (g, h]} = 1\} \\
 		&=\{d_{i, j}\ |\ i = g+1 \text{ and } j \notin(g, h]\} \cup \{d_{i, j}\ |\ i \in (g+1, h] \text{ and } j \notin(g, h]\} \\
		&= (\underbrace{\{d_{i, j}\ |\ \abs{\{i, j\} \cap (g, g+1]} = 1\}}_{\eqqcolon B_1} \setminus \underbrace{\{d_{i, j}\ |\ i = g+1 \text{ and } j \in (g+1, h]\}}_{\eqqcolon A})\\
		& \cup
			(\underbrace{\{d_{k, i}\ |\ \abs{\{k, i\} \cap (g+1, h]} = 1 \}}_{\eqqcolon B_2} \setminus \underbrace{\{d_{k, i}\ |\ k = g+1 \in  \text{ and } i \in (g+1, h]\}}_{= A}) \ .
	\end{align}
	Now, the sum over $B_1$ and $B_2$ is $D_{g, g+1}$ and $D_{g+1, h}$, respectively.
	The sum over $A$ is $\sum_{a = g+1}^h d_{g, a}$.
	Hence $D_{g, h} = D_{g, g+1} + D_{g+1, h} - 2 \sum_{a = g+1}^h d_{g, a}$.
	One can verify that $A_{g+1, h} = \sum_{a = g+1}^h d_{g, a}$, concluding the proof.
\end{proof}
The recursion in \cref{eq:recursive-dac} can be implemented using $\cO(n^2)$ time.
This is quite remarkable, as it means that in $\cO(n^2)$ time, we can either compute the demand across one cut, or the demands across all cuts.

Using \cref{eq:minimal-capacities}, Step 2 runs in $\cO(n^2)$.

In Step 3, all but at most $\lfloor \frac{n}{2} \rfloor$ demands are routed all front or all back.
The result of \cref{lemma:tight-cuts-remain-tight} is that we can find a tight cut for each edge in the beginning and reuse those, instead of searching for new tight cuts in each iteration.
Effectively, this means that in Step 3, all demands that are parallel to one of these initial tight cuts are routed all front or all back -- or at least could be without harm due to \cref{lemma:parallel-to-tight-cut}.
This observation also alleviates adjusting capacities and recomputing the demands across cuts in each iteration.
Altogether, we can replace Step 3 with the following subroutine:

\begin{algorithm}[H]
	\begin{mdframed}[backgroundcolor=green!10,linecolor=white,innerleftmargin=25pt,leftmargin=-25pt,rightmargin=15pt]
		$t[g] = h$, where $h \in [n]$ such that $\{g, h\}$ is a tight cut, for all $g \in [n]$\;
		$r[i] = i+1$ for all $i \in [n]$\;
		\For{$g \in [n]$}{
			\For{$i \in [n]$}{
				\While{$r[i] \neq i$ and $d_{i, r[i]}$ is parallel to $\{g, t[g]\}$}{
					 Route $d_{i, r[i]}$ to miss $\{g, h\}$\;
					 $r[i] = ((r[i] + 1) \mod n) + 1$\;
				}	
			}
		}
	\end{mdframed}
	\caption{A $\cO(n^2)$ subroutine replacing Step 3 of \cref{algo:rrl}.}
	\label{algo:step3}
\end{algorithm}
Here, the idea is that at all times, if $r[i] > j$, the demand $d_{i,j}$ (or $d_{j, i}$) has already been routed.
It can be shown by contraposition that if a demand $d_{i,j}$ is parallel to any of the initial tight cuts $\{g, h\}$, it is routed before or in the $g$-th iteration of the outer for-loop.
Also, one can show that all demands that are not parallel to any of the initial tight cuts are mutually crossing.
Hence the new subroutine computes an output with the same properties as Step~3 in \cref{algo:rrl}.
We furthermore observe that for each $i \in [n]$, $r[i]$ is increased at most $n$ times, implying the runtime is in $\cO(n^2)$.

In \cref{algo:rrl}, while the edge capacities must not be updated in each iteration of Step~3, they must be known before Step~4.
The residual capacities are simply the difference between the original capacities and the edge loads.
Again using a recursion, the edge loads resulting from the binary "pre-routing" obtained in Step~3 can be computed in $\cO(n^2)$ time.

\begin{lemma}
	\label{lemma:link-loads}
	For all $k \in [n]$ let $L_k^\mathrm{f}$ and $L_k^\mathrm{b}$ be the total traffic that is routed forward and backward, respectively, through the edge $\{k, k+1\}$ by the routing $\Phi$. 
	With $L_{-1}^\mathrm{b} \coloneqq L_n^\mathrm{b}$, the following holds true:
		\begin{align}
		L_k^\mathrm{f}(\Phi) &\coloneqq \begin{cases}
			\sum_{j = 2}^n d_{1,j} \cdot \Phi(1, j), & \text{if } i = 1 \\
			L_{i-1}^\mathrm{f} - \sum_{j = 1}^{i-1} d_{j,i} \cdot \Phi(j, i) + \sum_{j = i}^{n} d_{i,j} \cdot \Phi(i, j) & \text{if } i > 1
		\end{cases} \ , \\
		L_k^\mathrm{b}(\Phi) &\coloneqq \begin{cases}
			\sum_{i = 1}^n \sum_{j = i+1}^n d_{i,j} \cdot (1 - \Phi(i, j)) & \text{if } i = n\\
			L_{i-1}^\mathrm{b} - \sum_{j = i}^{n} d_{i, j} \cdot (1 - \Phi(i, j)) + \sum_{j = 1}^{i-1} d_{i,j} \cdot(1 - \Phi(i, j)) & \text{if } i < n
		\end{cases} \ .
	\end{align}
\end{lemma}

In Step~4 of \cref{algo:rrl}, the minimal slacks on the front routes of the remaining demands have to be determined.
Doing this the naïve way takes $\cO(n^2)$ time per demand, so $\cO(n^3)$ time in total (again utilizing \cref{eq:recursive-dac}).
This, too, can be improved to $\cO(n^2)$.
The precise way to do this cannot be discussed in this work. 
The interested reader is referred to the appendix.


%\section{Improvements to Upper and Lower Bounds}

Theorem \cref{theo:ring-loading-algorithm} guarantees that the solution provided by \cref{algo:rl} is not greater than $L^\mathrm{opt}\frac{3}{2} D$.
However, it is not clear that this upper bound is tight.
In this section we will examine this upper bound in greater detail.
Practically speaking, we are looking for the infimum $\alpha \in \R$ that satisfies the following expression:

\begin{equation}
	...
\end{equation}

\citet{skutella16} improved the upper bound to $\frac{19}{14} \approx 1.35$ and the lower bound to $1.1$.
\Todo{Maybe show skutella example for lower bound.}
Most recently, \citet{daubel19} further improved the upper bound to $1.3$.

\section{Conclusion}

In this work, the approximation algorithm for \RL by \citet{schrijver99} was presented.
On the way, we have seen how an exact solution to \RRL can be determined in $\cO(k n^2)$ time, where $k$ is the number of non-zero demands.
In this solution, all but at most $\lfloor \frac{n}{2} \rfloor$ demands are split.
These demands can be rerouted to obtain a binary routing which achieves a ringload of $L \leq \Lopt + \frac{3}{2}D$, where $\Lopt$ is the optimal binary ringload and $D$ the maximal demand.

Since its publication, Schrijver et al.'s  algorithm has further been used by \citet{khanna97} to obtain a polynomial-time approximation scheme for \RL.
Also, there is still active research on the algorithm's approximation quality:
It turns out that the constant $\frac{3}{2}$ in \cref{theo:ring-loading-algorithm} is only a rough upper bound.
The "best" upper bound $\alpha \in \R$ would be the infimum of all $\beta \in \R$ such that the following holds true:
\begin{quote}
	For all $n \in \N$, $x_i, y_i \in \R_{\geq 0}$ such that $x_i + y_i \leq 1$ for all $i \in [n]$, there exist $z_i \in \R_{\geq 0}$ such that for all $i \in [n]$
	\begin{equation}
		(z_i = y_i \quad \text{ or } \quad z_i = -x_i) 
		\quad \text{ and } \quad \abs{\sum_{i=1}^k z_i - \sum_{i = k+1}^n z_i} \leq \beta \quad \forall k \in [n] \ .
	\end{equation}
\end{quote}
The latest results on this problem have shown that $\alpha \leq \frac{13}{10}$ and $\alpha \geq \frac{11}{10}$ \cite{skutella16, daubel19}.

\clearpage          % neue Seite f�r Literaturverzeichnis

%%%%%%%%%%%%%%%%%%%%%%%%%%%%%%%%%%%%%%%%%%%%%%%%%%%%%%%%%%%%%%%%%%%%%%%%%%%%%%%%%%%%%%%%%%%%
% Literaturverzeichnis
%\nocite*  % Nicht zitierte Quellen werden auch ins Literaturverzeichnis aufgenommen
\thispagestyle{empty}
\bibliography{bibliography}
\bibliographystyle{plainnat}  % Literaturverzeichnis liegt in der Datei seminararbeit.bib

%%%%%%%%%%%%%%%%%%%%%%%%%%%%%%%%%%%%%%%%%%%%%%%%%%%%%%%%%%%%%%%%%%%%%%%%%%%%%%%%%%%%%%%%%%%%
%%%%%%%%%%%%%%%%%%%%%%%%%%%%%%%%%%%%%%%%%%%%%%%%%%%%%%%%%%%%%%%%%%%%%%%%%%%%%%%%%%%%%%%%%%%%
% Ende des Dokuments
\end{document}			
