\begin{figure}[ht]
	\vspace*{-1em}
	\centering
%	\DontPrintSemicolon
	\begin{algorithm}[H]
		\begin{mdframed}[backgroundcolor=green!10,linecolor=white,innerleftmargin=25pt,leftmargin=-25pt,rightmargin=15pt]
			\KwIn{Ring size $n \in \N$, demands $d_{i,j}$, $1 \leq i < j \leq n$.}
			\nl Compute demands across cuts $D_{i, j}$ and $M = \max_{1 \leq i < j \leq n} D_{i,j}$\;
			\nl Compute the capacities $C_k$ in \cref{eq:minimal-capacities}\;
			\nl\While{there exist parallel demands $d_{i,j}, d_{u, v}$}{
				Choose edge $\{g, g+1\}$ that lies in between $d_{i,j}$ and $d_{u, v}$\;
				Find a tight cut $\{g, h\}$\;
				Route $d_{i, j}$ or $d_{u, v}$ to miss the cut $\{g, h\}$\;
				Decrease capacities accordingly and adjust demands across cuts\;
			}
			\nl\ForEach{unrouted demand $d_{i, j}$}{
				Compute minimal slack $\mu$ on the front route of $d_{i, j}$\;
				Route $\min(d_{i, j}, \frac{\mu}{2})$ forward and the remainder backwards\;
				Decrease capacities accordingly and adjust demands across cuts\;
			}
%			\nl\Return{Routing $\Phi$}
		\end{mdframed}
		\caption{Schrijver et al.'s \cite{schrijver99} algorithm for computing a minimal solution to an instance of \RRL.}
		\label{algo:rrl}
	\end{algorithm}
\end{figure}