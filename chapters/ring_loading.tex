\section{An Algorithm for Ring Loading}

The idea for solving an instance of \RL is the following:
First, we compute a solution to the same instance in \RRL.
As \cref{lemma:number-of-split-demands} states, \cref{algo:rrl} yields a solution $\phi$ in which at most $\lfloor\frac{n}{2}\rfloor$ demands are split; the remainder is either routed all front or all back.
Now, we iteratively unsplit those demands.
Of course, this does not always yield the optimal solution $L^\mathrm{opt}$ to the ring loading instance.
However, the way in which we unsplit the demands approximates $L^\mathrm{opt}$ with an additive error of at most $\frac{3}{2}D$, where $D$ is the largest demand.

The procedure for unsplitting is given in \cref{algo:rl}.

\begin{figure}[ht]
	\begin{algorithm}[H]
		\KwData{Ring size $n \in \N$, demands $d_{ij}$, $1 \leq i < j \leq n$.}
		Let $\phi$ be solution to the same instance in \RRL obtained using \cref{algo:rrl} and $S = \{(a_1, b_1), \ldots (a_m, b_m)\}$ the set of indices of split demands\;
		$D = \max_{1 \leq i < j \leq n} D_{ij}$\;
		$s = 0$\;
		\For{$i =1, \ldots, m$}{
			\If{$s+\phi(a_i, b_i) \leq \frac{D}{2}$}{
				$s = s + \phi(a_i, b_i)$\;
				$\phi(a_i, b_i) = 1$\;
			}
			\Else{
				$s = s - (1 - \phi(a_i, b_i))$\;
				$\phi(a_i, b_i) = 0$\;
			}
		}
%		Initialize $\theta_G$ and $\theta_D$\;
%		\While{not converged}{
%			\For{$i=1, \ldots, k$}{
%				Sample batch of $B$ samples $z_1, \ldots, z_B$ from $p_z$\;
%				Sample batch $x_{i_1}, \ldots, x_{i_B}$ from the training data\;
%				Compute the stochastic gradient 
%				$$\nabla_{\theta_D} \frac{1}{B} \sum_{j=1}^{B} \left( \ln D(x_{i_j}) + \ln(1 - D(G(z_j))) \right);$$
%				Update $\theta_D$ by ascending the gradient according to the learning rule $R$\;
%			}
%			Sample batch of $B$ samples $z_1, \ldots, z_B$ from $p_z$\;
%			Compute the stochastic gradient
%			$$\nabla_{\theta_G} \frac{1}{B} \sum_{j=1}^{B} -\ln(D(G(z_j)));$$
%			Update $\theta_G$ by descending the gradient according to the learning rule $R$\;
%		}
		\caption{Algorithm for \RL}
		\label{algo:rl}
	\end{algorithm}
\end{figure}

The following theorem and its proof guarantee the quality of the produced solution.
\begin{theorem}
	\label{theo:ring-loading-algorithm}
	Let $I$ be an instance of size $n$ of \RL, $D$ its largest demand and $L^\mathrm{opt}$ its optimal solution.
	Then \cref{algo:rl} computes a solution to $I$ with maximum load $L \leq L^\mathrm{opt} + \frac{3}{2} D$ in $\cO(?)$ time using $\cO(?)$ space.
\end{theorem}
\begin{proof}
	TBD
\end{proof}

