\section{An Algorithm for Ring Loading}
\label{sec:ring-loading}

In this section, we see how we can transform the real-valued routing $\Phi$ determined by \cref{algo:rrl} into a binary routing that approximates the optimal routing.
Let $L_i = L_i(\Phi)$ be the edges loads induced by $\Phi$ and $S$ be the set of demands that are split.
In order to obtain a binary routing, we simply have to reroute the demands in $S$ all front or all back -- ideally without increasing the maximal edge load by too much. 
As we have seen, all demands in $S$ are mutually crossing and $\abs{S} \leq \lfloor \frac{n}{2} \rfloor$.
That means, each node is an endpoint of at most one demand.
For simplicity, we remove all nodes $i \in [n]$ that are not endpoints of any demand.
When rerouting any split demand, the loads on the two edges incident to $i$ change by the same amount.
Hence, we replace the edges $\{i-1, i\}$ and $\{i, i+1\}$ with a single new edge $\{i', i'+1\}$ and set 
\begin{equation}
	L_{i'} \coloneqq \max(L_{i-1}, L_i) \ .
\end{equation}
For rerouting the split demands, it suffices to consider this contracted instance $I'$ of size $2m$, where $m \coloneqq \abs{S}$.
We can rewrite $S$ in a simpler way as $S' = \{d_{i, i+m}\ |\ i \in [m]\}$.

For all $d_{i, i+m} \in S'$, let $x_i$, $y_i$ be the amount of traffic of $d_{i, i+m}$ that $\Phi$ routes through the front and the back, respectively.
Rerouting $d_{i, i+m}$, e.g. by routing it all front increases the the edge loads on the front route of $d_{i, i+m}$ by $y_i$, while those on the back route are decreased by $y_i$.
Similarly, routing $d_{i, i+m}$ all back increases the loads on the back route and decreases those in the front by $x_i$, respectively.

Now, assume that all demands in $S'$ have been routed either all front or all back.
For each $d_{i, i+m} \in S'$ we define
\begin{equation}
	z_i = \begin{cases}
		y_i, &\text{if } d_{i, i+m} \text{ is routed all front} \\
		-x_i, &\text{if } d_{i, i+m} \text{ is routed all back}
	\end{cases} \ ,
\end{equation}
that is, $z_i$ is the change of traffic on front route of $d_{i, j}$.
This allows us to formulate the total change of traffic on any edge $\{k, k+1\}$, $k \in [2m]$ and determine the new link load $L_k'$:
\begin{equation}
	\label{eq:load-change}
	L_k' = L_k + \sum_{i \in [m], k \in [i, i+m)} z_i - \sum_{i \in [m], k \notin [i, i+m)} z_i \ .
\end{equation}
Here, we have $k \in [i, i+m)$ if and only if the edge $\{k,k+1\}$ lies on the front route of $d_{i, i+m}$.


The following theorem shows that there is a way to reroute the demands in $S'$ such that the maximal edge load remains bounded.
\begin{lemma}
	\label{lemma:reroute-demands}
	Let $x_i, y_i \in \R_{\geq 0}, i \in [m]$ be such that $x_i + y_i \leq D$ for $D \in \R_{\geq 0}$.
	Then there exist $z_i \in \R$, $i \in [m]$ such that:
	\begin{align}
		(z_i = y_i \quad \text{ or } \quad z_i = -x_i) 
		\quad \text{ and } \quad \sum_{i=1}^k z_i \in \left[-\frac{D}{2}, \frac{D}{2}\right] \quad \forall k \in [m] \ .
	\end{align}
\end{lemma}
\begin{proof}
	We iteratively define the $z_i$.
	Set $z_1 \coloneqq y_1$.
	Then, for $k \in [n]$, let $z_i, 1 \leq i < k$ be such that $\sum_{i=1}^{k-1} z_i \in \left[-\frac{D}{2}, \frac{D}{2}\right]$.
	Since $x_k + y_k \leq D$, at least one of the inequalities $\sum_{i=1}^{k-1} z_i + y_k \leq \frac{D}{2}$, $\sum_{i=1}^{k-1} z_i - x_i \geq -\frac{D}{2}$ holds true.
	Choose $z_k = y_k$ or $z_k = -x_k$ accordingly.	
\end{proof}
\begin{theorem}
	\label{theo:ring-loading-algorithm}
	Let $\Phi$ be the routing obtained from \cref{algo:rrl} and $\Lopt$ be the ringload of an optimal binary routing.
	Then the demands that are split by $\Phi$ can be rerouted such that the resulting binary routing $\Phi'$ has a ringload of
	\begin{equation}
		L(\Phi') \leq \Lopt + \frac{3}{2}D \ ,
	\end{equation}
	where $D = \max_{1 \leq i < j \leq n} d_{i, j}$ is the largest demand.
\end{theorem}
\begin{proof}
	For all $d_{i, i+m} \in S'$, let $x_i$, $y_i$ be the amount of traffic of $d_{i, i+m}$ that $\Phi$ routes through the front and the back, respectively.
	Choose $z_i, i \in [m]$ like in \cref{lemma:reroute-demands} with $D = \max_{1 \leq i < j \leq n} d_{i, j}$.
	We construct the binary routing $\Phi'$ by setting $\Phi'(i, j) \coloneqq \Phi(i, j)$ for all $d_{i,j} \notin S$.
	For all demands $d_{i, j}$ that are split by $\Phi$, we set
	\begin{equation}
		\Phi'(i, j) = \begin{cases}
			1, & \text{if } z_i = y_i \\
			0, & \text{if } z_i = -x_i
		\end{cases}
	\end{equation}
	where $z_i$ corresponds to the demand $d_{i, j}$ in $S$.
	We get:
	\begin{align}
		L(\Phi') - L(\Phi) &= \max_{k \in [n]} (L_k(\Phi') - L(\Phi)) 
		\leq \max_{k \in [n]} (L_k(\Phi') - L_k(\Phi))\\
		& \stackrel{\ref{eq:load-change}}{=} \max_{k \in [m]} \left(\sum_{i \in [m], k \in [i, i+m)} z_i - \sum_{i \in [m], k \notin [i, i+m)} z_i\right)\\
		&= \max_{k \in [m]} \abs{\sum_{i = 1}^k z_i - \sum_{i = k+1}^m z_i }\\
		&= \max_{k \in [m]} \abs{2\sum_{i = 1}^k z_i - \sum_{i = 1}^m z_i } \ . \label{eq:partial-reroutes}
	\end{align}
	Because both $\sum_{i = 1}^k z_i$ and $\sum_{i = 1}^m z_i$ lie in the interval $[-\frac{D}{2}, \frac{D}{2}]$, 
	we have 
	\begin{equation}
		2\sum_{i = 1}^k z_i - \sum_{i = 1}^m z_i \in [-\frac{3}{2}D, \frac{3}{2}D] \ .
	\end{equation}
	Thus we get $L(\Phi') - L(\Phi) \leq \frac{3}{2}D$, or equivalently $L(\Phi') \leq L(\Phi) + \frac{3}{2}D$.
	Observing that $L(\Phi) \leq \Lopt$ concludes the proof.
\end{proof}