\section{Introduction}

In this work, the 1999 paper "The Ring Loading Problem" by Schrijver and Seymour will be presented and discussed.
This will be the introduction.

This work is based on \citet{schrijver99}.

Already shown: Tight bound $L^\mathrm{opt} \leq 2 L^\ast$ (take references from däubel intro!! TODO).

\citet{schrijver99} achieve an additive bound of $L^\mathrm{opt} + \frac{3}{2}D$.
This bound has since been improved to $1.3$ but the improvement based on the same algorithm.
It is therefore still sensible to review and understand the algorithm presented by \citet{schrijver99} in the late 1990s.

The authors of \cite{schrijver99} claim that their algorithm runs in $O(k n^2)$, where $k$ is the number of non-zero demands.

In this work, I show that the ideas presented by \citet{schrijver99} lead to an algorithm that runs in $\cO(n^2)$.
This is almost optimal, as the encoding of an instance is in $\Theta(k)$ and in the worst case $k = \binom{n}{2} \in \cO(n^2)$.
This algorithm fist computes a solution to \RRL, which is then transformed into an approximate solution to the original \RL problem.

This seems to be a novel result which is interesting for several reasons:
\begin{itemize}
	\item The best currently known runtime for computing an $L^\mathrm{opt} + \frac{3}{2}D$ approximation to \RL is $O(k n^2)$.
	The proposed algorithm improves this runtime by a factor of $k$.
	\item The best currently known algorithm for solving \RRL with integer demands is $O(k + t_k)$ for integer demands, where $t_k$ is the time for sorting $k$ integers.
	For real demands, the best known runtime is $O(\min\{kn, n^2\})$.
	The proposed algorithm also works for real demands and almost matches the runtime of the latter algorithm.
	\item The proposed algorithm computes the same solution to \RL and \RRL as Schrijver et al.'s algorithm, which is particularly aesthetic in the sense that at most $\lfloor\frac{n}{2}\rfloor$ demands are split while the others are routed all front or all back.
	This means that the runtimes of all algorithms that use Schrijver et al.'s algorithm as a subroutine are improved as well, in particular that of the $(1 + \epsilon)$ algorithm by \citet{khanna97}.
\end{itemize}
