\section{Conclusion}

In this work, the approximation algorithm for \RL by \citet{schrijver99} was presented.
On the way, we have seen how an, exact solution to \RRL can be determined in $\cO(k n^2)$ time, where $k$ is the number of non-zero demands.
In this solution, all but at most $\lfloor \frac{n}{2} \rfloor$ demands are split.
These are rerouted in a way to achieve a ringload of $L \leq \Lopt + \frac{3}{2}D$, where $\Lopt$ is the optimal binary ringload and $D$ the maximal demand.
Since its publication, the algorithm has further been used by \citet{khanna97} to obtain a polynomial-time approximation scheme for \RL.

Also, there is still active research on the approximation quality of Schrijver et al.'s algorithm:
It turns out that the constant $\frac{3}{2}$ is only a rough upper bound.
The "best" upper bound $\alpha \in \R$ would be the infimum of all $\beta \in \R$ such that the following holds true:
\begin{quote}
	For all $n \in \N$, $x_i, y_i \in \R_{\geq 0}$ such that $x_i + y_i \leq 1$ for all $i \in [n]$, there exist $z_i \in \R_{\geq 0}$ such that for all $i \in [n]$
	\begin{equation}
		(z_i = y_i \quad \text{ or } \quad z_i = -x_i) 
		\quad \text{ and } \quad \abs{\sum_{i=1}^k z_i - \sum_{i = 1}^n z_i} \leq \beta \quad \forall k \in [n] \ .
	\end{equation}
\end{quote}
The latest results on this problem have shown that $\alpha \leq \frac{13}{10}$ and $\alpha \geq \frac{11}{10}$ \cite{skutella16, daubel19}.